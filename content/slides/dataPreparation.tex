%%%%%%%%%%%%%%%%%%%%%%%%%%%%%%
\section{Preparación y transformación de datos}
%%%%%%%%%%%%%%%%%%%%%%%%%%%%%%

\begin{frame}{Preparación y transformación de datos}
 \begin{columns}[T]
    \begin{column}{.75\textwidth}
     \begin{wideitemize}
      \item Limpieza de datos.
      \item Datos no disponibles.
      \begin{itemize}
      \item Gestión de valores vacíos.
      \item Imputación de datos no disponibles.
      \end{itemize}

      \item Transformación de datos.
      \begin{itemize}
      \item \textit{Data munging} o \textit{data wrangling}.
      \item Pasar los datos a otro formato o dejarlos preparados para luego poder 
      analizarlos más fácilmente.
      \end{itemize}

    \end{wideitemize}
    \end{column}
    \begin{column}{.25\textwidth}
    \vspace*{1cm}
    \includegraphics[width=1.2\textwidth]{figs/liftarn-Cleaning-tools.png}
    \end{column}
  \end{columns}

\end{frame}

%%---------------

\begin{frame}{Data Wrangler}
\begin{itemize}
 \item Ejemplo de este tipo de herramientas (UW Interactive Data Lab).
\end{itemize}

\begin{figure}
 \centering
 \includegraphics[width=1\textwidth]{figs/data-wrangler.jpeg}
\end{figure}

\end{frame}

%%---------------

\begin{frame}{Preparación de datos}
 \begin{columns}[T]
    \begin{column}{.75\textwidth}
     \begin{wideitemize}
      \item En primer lugar, debemos comprobar que no existen valores extraños
      ni datos omitidos.
      
      \begin{itemize}
       \item Utilizar técnicas básicas de resumen de datos.
       \item Técnicas de visualización de datos omitidos.
      \end{itemize}
      
      \item Después, tenemos dos opciones:
      \begin{itemize}
       \item Descartar los casos que contengan variables con datos omitidos.
       \item Imputar valores para los datos que faltan, utilizando técnicas
       avanzadas de imputación de múltiples valores.
      \end{itemize}

    \end{wideitemize}
    
    \end{column}
    \begin{column}{.25\textwidth}
    \vspace*{1cm}
    \includegraphics[width=1.2\textwidth]{figs/liftarn-Cleaning-tools.png}
    \end{column}
  \end{columns}

\end{frame}

%%---------------

\begin{frame}{Transformación de datos}
 \begin{columns}[T]
    \begin{column}{.75\textwidth}
     \begin{wideitemize}
      \item Otro paso crucial antes de comenzar nuestro análisis es comprobar
      la distribución de valores de los parámetros implicados.
      
      \begin{itemize}
       \item Muchas técnicas y modelos asumen que los datos siguen una cierta
       distribución (e.g. Normal), pero puede no ser cierto.
       
       \item De hecho, en la práctica nos encontramos muchas veces con distribuciones
       sesgadas (\textit{skewed distributions}) o con diferentes apuntamientos
       (\textit{kurtosis}).
      \end{itemize}
      
    \item Posibles objetivos:
    \begin{itemize}
     \item Reducir la asimetría de la distribución de valores.
     
     \item Transformar una o varias variables de forma que se parezcan más a una
     distribución Normal (univariante o multivariante).
    \end{itemize}

    \end{wideitemize}
    
    \end{column}
    \begin{column}{.25\textwidth}
    \vspace*{1cm}
    \includegraphics[width=1\textwidth]{figs/primary-transform.png}
    \end{column}
  \end{columns}

\end{frame}

%%---------------

\begin{frame}[fragile]{Aspectos adicionales}
 \begin{columns}[T]
    \begin{column}{.75\textwidth}
     \begin{wideitemize}
      \item Tranformación entre diferentes formatos de datos
      \begin{itemize}
       \item \textit{Wide format} vx. \textit{long format}.
      \end{itemize}

    \end{wideitemize}
    
    \begin{footnotesize}
    \begin{verbatim}
# Wide format
subject sex control cond1 cond2
       1   M     7.9  12.3  10.7
       2   F     6.3  10.6  11.1
       3   F     9.5  13.1  13.8
       4   M    11.5  13.4  12.9
    \end{verbatim}
    \end{footnotesize}
    
    \end{column}
    \begin{column}{.25\textwidth}
    \vspace*{1cm}
    \includegraphics[width=1\textwidth]{figs/primary-transform.png}
    \end{column}
  \end{columns}

\end{frame}

%%---------------

\begin{frame}[fragile]{Aspectos adicionales}
 \begin{columns}[T]
    \begin{column}{.75\textwidth}
        
    \begin{footnotesize}
    \begin{verbatim}
# Long format
subject sex condition measurement
       1   M   control         7.9
       1   M     cond1        12.3
       1   M     cond2        10.7
       2   F   control         6.3
       2   F     cond1        10.6
       2   F     cond2        11.1
       3   F   control         9.5
       3   F     cond1        13.1
       3   F     cond2        13.8
       4   M   control        11.5
       4   M     cond1        13.4
       4   M     cond2        12.9 
    \end{verbatim}
    \end{footnotesize}
    
    \end{column}
    \begin{column}{.25\textwidth}
    \vspace*{1cm}
    \includegraphics[width=1\textwidth]{figs/primary-transform.png}
    \end{column}
  \end{columns}

\end{frame}

%%---------------