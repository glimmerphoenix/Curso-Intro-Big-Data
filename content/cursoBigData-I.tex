\documentclass[10pt]{beamer}
%\usetheme{default}
%\usetheme{Warsaw}
%\usetheme{boxes}
%\usetheme{Pittsburgh}
\usetheme{DEIOURJC}
 
\mode<presentation>


\usepackage[english]{babel}
\usepackage[utf8]{inputenc}
\usepackage{etex}

\usepackage{listings}
\usepackage{times}
\usepackage[T1]{fontenc}
% Or whatever. Note that the encoding and the font should match. If T1
% does not look nice, try deleting the line with the fontenc.

%\usepackage{beamerthemesplit}
%\usepackage{graphicx,times,/papers/styles/psfig,amsmath} % Add all your packages here
\usepackage{graphicx,times,amsmath} % Add all your packages here
\usepackage{pstricks,pst-node,pst-text,pst-3d}
\usepackage{amsmath}
\usepackage{amsfonts}
  \newcommand{\field}[1]{\mathbb{#1}}
\usepackage{epsf}
\usepackage{algorithm,algorithmic}

% Hyperlinks
\usepackage{hyperref}
\usepackage{url}

% Redefine font size for captions
\setbeamerfont{caption}{size=\footnotesize}
\usepackage[labelsep=endash, figurename=Fig.]{caption}

\DeclareGraphicsExtensions{.pdf}

\newcommand{\OO}{{\rm O}}

%Espacio extra entre líneas itemize
\newenvironment{wideitemize}{\itemize\addtolength{\itemsep}{8pt}}{\enditemize}

%%%%%%%%%%%%%%%%%%%%%%%%%%%%%
\title[Big Data I: Ingeniería de datos]
{Big Data I:\\~\\
Ingeniería de datos}

% \author[Felipe Ortega, Javier M. Moguerza]{
% Felipe Ortega, Javier M. Moguerza\\
% Dpto. de Estad\'{i}stica e Investigaci\'{o}n Operativa\\
% Universidad Rey Juan Carlos
% }

\author[Felipe Ortega]{
Felipe Ortega\\
Dpto. de Estadística e Investigación Operativa\\
Universidad Rey Juan Carlos
}

\date{\today}

% Seleccionar archivo a compilar
\includeonly{./slides/intro}

\begin{document}

% ESTO ES PARA QUE NO HAYA footline EN LA titlepage
{\setbeamertemplate{footline}{} 
\begin{frame}
  \titlepage
\end{frame}
} \addtocounter{framenumber}{-1}

\begin{frame}{}
 \vspace{3cm}
 \begin{flushright}
 \includegraphics[width=0.3\textwidth]{figs/by-sa.png}\\
    \begin{small}(cc)2015 Felipe Ortega.\\
    Algunos derechos reservados.\\
    Este documento se distribuye bajo una licencia
    Creative Commons Reconocimiento-CompartirIgual 4.0,
    disponible en:
    \url{http://creativecommons.org/licenses/by-sa/4.0/es/} 
    \end{small}
 \end{flushright}
 
\end{frame}

% TABLE OF CONTENT
% \begin{frame}{Contenidos}
%   \tableofcontents
% \end{frame}

% MATRIZ DE CONTENIDOS INICIALES
% Curso de 0.1 créditos para la ETSIT (9 abril 2014)
% \include{slides/intro-big-data.tex}

% Curso ETSII de 1 crédito.
% Big data I: Introducción a la ingeniería de datos (marzo-abril 2015)

% Introducción a la ingeniería de datos
%% Big Data I: Ingeniería de datos
%
% Felipe Ortega, Javier M. Moguerza
% DEIO, URJC.
%
%
%%%%%%%%%%%%%%%%%%%%%%%%%%%%%%
\section{Introducción a ingeniería de datos}
%%%%%%%%%%%%%%%%%%%%%%%%%%%%%%

\begin{frame}{}
\begin{center}
 \huge Introducción a la ingeniería de datos
\end{center}
\end{frame}

%%---------------

\begin{frame}{Objetivos del curso}
\begin{wideitemize}
 \item Introducción a la metodología, aspectos técnicos y de infraestructura para 
 ingeniería de datos.
 \item  Fundamentos para comprender el papel y la importancia  de los métodos
 y tecnologías de ingeniería de datos en la actualidad.
 \item Ilustraremos con numerosos ejemplos tecnológicos y casos de estudio.
 \item Conoceremos tendencias actuales en ingeniería de datos e infraestructuras
 asociadas.
\end{wideitemize}
\end{frame}

%%---------------

\begin{frame}{¿Qué es la ciencia de datos?}
\begin{figure}[h]
\centering
\includegraphics[width=0.5\textwidth]{figs/data-science-venn-diagram.jpeg} 
\caption{\href{http://drewconway.com/zia/2013/3/26/the-data-science-venn-diagram}
{Diagrama de Venn de la Ciencia de Datos (por Drew Conway)}.}
\end{figure}
 
\end{frame}

%---------------

\begin{frame}{Data-intensive science}
 \begin{wideitemize}
  \item Avances científicos fundamentados en el análisis de grandes y complejos volúmenes de
  datos, posibilitados por los avances tecnológicos en computación y los métodos
  de estudio [1].
  
  \item Aparece como evolución de los 3 paradigmas científicos anteriores:
  \begin{enumerate}
   \item Ciencia empírica.
   \item Ciencia teórica.
   \item Ciencia computacional.
  \end{enumerate}


 \end{wideitemize}

\end{frame}

%%---------------

\begin{frame}{Ciencia de datos: multidisciplinariedad}
 \begin{wideitemize}
  \item Este solape entre diferentes disciplinas también sugiere que será muy
  complicado encontrar una sola persona que acumule todo el conocimiento necesario
  para realizar este trabajo con garantías:
  \begin{itemize}
   \item Matemáticas.
   \item Estadística.
   \item Computación.
   \item Desarrollo de software.
   \item Data mining/machine learning.
   \item Comunicación.
   \item Visualización de datos.
   \item Experiencia en el área de negocio/aplicación.
  \end{itemize}

  \item La única solución es contar con equipos de trabajo \textbf{multidisciplinares}.

 \end{wideitemize}

\end{frame}

%%---------------

\begin{frame}{Ingeniería + análisis de datos}
 \begin{figure}
\centering
\includegraphics[width=0.99\textwidth]{figs/ingdatos-workflow.png} 
\end{figure}
Basado en Fig 1-1 de [2].

\end{frame}

%---------------

% ---------------

\begin{frame}{Tareas en ingeniería de datos}
 \begin{columns}[T]
    \begin{column}{.8\textwidth}
  \begin{wideitemize}
  \item Obtención de datos.
  \begin{itemize}
   \item Gestión de múltiples fuentes de datos (offline vs. tiempo real).
  \end{itemize}

  \item Almacenamiento de datos.
  \begin{itemize}
   \item Datos estructurados vs. no estructurados.
   \item Datos enlazados.
   \item Metadatos y estándares de representación.
  \end{itemize}

  \item Preparación de datos.
  \begin{itemize}
   \item Limpieza de datos.
   \item Datos no disponibles (imputación).
  \end{itemize}
 \end{wideitemize}
    \end{column}
    \begin{column}{.20\textwidth}
    \vspace*{1.5cm}
    \includegraphics[width=0.9\textwidth]{figs/checklist.png}
    \end{column}
  \end{columns}
  
\end{frame}

%%---------------

\begin{frame}{Tareas en ingeniería de datos}
 \begin{wideitemize}
  \item Tratamiento de datos.
  \begin{itemize}
   \item Organización de conocimiento (ontologías).
   \item Identificación/extracción de datos relevantes.
  \end{itemize}

  \item Cómputo y paralelización.
  \begin{itemize}
   \item Particionado y compresión de datos.
   \item Multiprocesado y procesamiento paralelo (clusters, cloud computing).
   \item Paradigmas de cómputo (ej. Map Reduce).
  \end{itemize}
 \end{wideitemize}
 
\end{frame}

%%---------------

\begin{frame}{Tareas en ingeniería de datos: otros aspectos}
 \begin{wideitemize}
  \item Tecnologías y recursos de computación.
  \begin{itemize}
   \item Necesidad de adquirir nociones sobre el impacto de diferentes alternativas
   sobre el rendimiento de la infraestructura de computación.
   \item Planificación estratégica de uso de recursos.
  \end{itemize}

  \item Desarrollo y gestión de software.
  \begin{itemize}
   \item El código se convierte en activo fundamental.
   \item Importancia del software libre como opción preferencial para análisis
   de datos.
  \end{itemize}

  \item Gestión de datos.
  \begin{itemize}
   \item Mantener nuestros datos organizados, organización y aprovechamiento de
   metadatos (datos acerca de los datos).
  \end{itemize}
 \end{wideitemize}

\end{frame}

%%---------------

\begin{frame}{Data mining/machine learning}
 \begin{wideitemize}
  \item \textbf{Data mining}: Intentamos descubrir patrones o información que están aparentemente
  ocultas en los datos.
  \item \textbf{Machine learning}: Usamos los datos para entrenar algoritmos que luego
  realizarán tareas de forma automática (e.g. clasificación).
  \begin{itemize}
   \item \textit{Métodos supervisados}: Se proporcionan un listado de clases o grupos
   a priori, basado en el criterio de expertos (se supervisa el proceso).
   \item \textit{Métodos no supervisados}: No se proporciona de antemano información sobre
   los grupos o clases, sino que se espera encontrarlos de forma natural en los
   datos.
  \end{itemize}

 \end{wideitemize}

\end{frame}

%%---------------

\begin{frame}{Clasificación de tipos de problemas}
 \begin{wideitemize}
  \item Podemos identificar una serie de \textbf{problemas típicos} asociados al análisis
  de datos [2].
  
  \item 1. \textbf{Clasificación y estimación de probabilidades}: Intentamos predecir para
  cada elemento o individuo en un grupo a qué clase pertenece, de entre un conjunto
  finito de clases previamente establecidas (y con frecuencia, mutuamente excluyentes).
  
  \item 2. \textbf{Estimación/predicción de valores}: Creamos modelos estadísticos que nos
  permitan estimar el valor de una o varias variables de interés que describen a un
  elemento o individuo, o bien predecir su valor futuro.

 \end{wideitemize}

\end{frame}

%%---------------

\begin{frame}{Clasificación de tipos de problemas}
 \begin{wideitemize}
  \item 3. \textbf{Patrones de similitud}: Intentamos identificar elementos o individuos similares
  a uno ya dado, basado en la información descriptiva que tenemos sobre ellos.
  \begin{itemize}
   \item Ejemplo: empresa interesada en descubrir otras compañías similares a
   sus mejores clientes para aumentar su cuota de mercado.
  \end{itemize}
  
  \item 4. \textbf{Clustering} (conglomerados): Intentamos agrupar individuos o elementos en
  grupos basándonos en criterios de similitud, pero sin un propósito inicial.
  \begin{itemize}
   \item Ejemplo: ¿Podemos agrupar los clientes de nuestra compañía en grupos
   o segmentos con similares características?
  \end{itemize}
  
  \item 5. \textbf{Co-ocurrencia}: Intentamos encontrar asociaciones entre entidades o
  individuos basándonos en sucesos o transacciones en las que están involucrados.
  \begin{itemize}
   \item Ejemplo: ``Los clientes que compraron el producto X también compraron...''.
  \end{itemize}

 \end{wideitemize}

\end{frame}

%%---------------

\begin{frame}{Clasificación de tipos de problemas}
 \begin{wideitemize}
  \item 6. \textbf{Profiling}: Caracterización del comportamiento típico de un
  individuo, grupo o población.
  \begin{itemize}
   \item Ejemplo: Patrones habituales de uso de las personas que poseen un smartphone.
  \end{itemize}
  
  \item 7. \textbf{Predicción de enlaces}: Se pretende descubrir potenciales
  nuevas conexiones entre los elementos que pertencen a una red (grafo o digrafo).
  \begin{itemize}
   \item Ejemplo: ``Puede que conozcas también a los siguientes amigos y quieras
   agregarlos a tu red...''.
  \end{itemize}
  
  \item 8. \textbf{Reducción de datos}: Tranformamos un conjunto de datos grande
  o con muchas dimensiones en otro más manejable, pero que siga siendo descriptivo
  respecto al proceso o fenómeno que estamos estudiando.
  \begin{itemize}
   \item Ejemplo: análisis de componentes principales.
  \end{itemize}

 \end{wideitemize}

\end{frame}

%%---------------

\begin{frame}{Clasificación de tipos de problemas}
 \begin{wideitemize}
  \item 9. \textbf{Causalidad}: Comprender qué eventos, acciones o factores
  influyen sobre un fenómeno de interés.
  \begin{itemize}
   \item Ejemplo: relación entre el consumo de tabaco y la aparación de ciertos
   tipos de tumores.
   \item Más complicado de demostrar de lo que podemos imaginar a priori.
  \end{itemize}

 \end{wideitemize}

\end{frame}

%%---------------

\begin{frame}{DDD: Data-Driven Decision-making}
 \begin{wideitemize}
  \item Cuidado con los \textbf{riesgos}:
  \begin{quotation}
``Puesto que podemos descubrir información y conocimiento directamente en
los datos, puede surgir la tentación de confiarnos ciégamente a los resultados
que nos ofrezcan las máquinas que ejecutan estos algoritmos''.
  \end{quotation} 
  
  \item Solución: la \textbf{toma de decisiones} se debe hacer basada en
  \textbf{evidencias} empíricas (\textit{data-driven}, \textit{evidence-based})...
  
  \item ...pero también necesitamos \textbf{interpretar} los resultados basándonos en la
  \textbf{experiencia} sobre un área de aplicación.
  
  \begin{itemize}
   \item Ejemplo: métodos bayesianos permiten incluir conocimiento o teorías previas
   al cálculo de nuestros modelos (\textit{prior distributions}).
  \end{itemize}

  
  \item No vale para ``echar la culpa a los datos o al análisis'' si la decisión 
  fue incorrecta.

 \end{wideitemize}

\end{frame}

%%---------------

%%%%%%%%%%%%%%%%%%%%%%%%%%%%%%
\section{Replicabilidad}
%%%%%%%%%%%%%%%%%%%%%%%%%%%%%%

\begin{frame}{}
\begin{center}
 \huge Replicabilidad en análisis de datos
\end{center}
\end{frame}

%---------------

\begin{frame}{Replicabilidad: elementos}
 \begin{wideitemize}
  \item Conjuntos de \textbf{datos} que se han utilizado.
  \item \textbf{Infraestructura} equivalente (recursos computacionales).
  \item \textbf{Software}:
  \begin{itemize}
   \item \textbf{Código} para llevar a cabo el análisis.
   \item \textbf{Dependencias} satisfechas (otros programas, bibliotecas, S.O., etc.).
   \item \textbf{Configuración} original para el análisis.
  \end{itemize}
  
  \item Metodología.
  \begin{itemize}
   \item Explicación detallada del \textbf{proceso} (limpieza y preparación de
   datos, análisis, resultados, conclusiones).
  \end{itemize}

 \end{wideitemize}

\end{frame}

%%---------------

\begin{frame}{Replicabilidad: workflow}

\begin{figure}
 \centering
 \includegraphics[width=0.95\textwidth]{figs/replica-workflow.jpeg}
\end{figure}

\end{frame}

%%---------------

\begin{frame}{Espectro niveles de replicación}

\begin{figure}
 \centering
 \includegraphics[width=0.95\textwidth]{figs/espectro-replica.jpeg}
\end{figure}

\end{frame}

%%---------------

\begin{frame}{Ejemplos análisis no replicables}
 \begin{wideitemize}
  \item \textbf{Oncología} [3]: Dpto. Biotecnología de la firma Amgen (Thousand 
  Oaks) sólo confirmó 6 de un total de 53 artículos emblemáticos. Bayer 
  HealthCare (Alemania) pudo validar un 25\% de estudios.
  \item \textbf{Psicología} [4]: De un total de 249 artículos de la APA, el 73\% 
  de los autores no respondieron sobre sus datos en 6 meses.
  \item \textbf{Economía y finanzas} [5]: Diferentes paquetes software producen 
  resultados muy distintos con técnicas estadísticas directas aplicadas sobre 
  datos idénticos a los originales.
 \end{wideitemize}

\end{frame}

%%---------------

\begin{frame}{Control de versiones}
 \begin{wideitemize}
  \item Herramientas avanzadas de gestión de código software.
  \item Ejemplos: Git, Mercurial.
  \begin{itemize}
   \item Desarrollo distribuido y altamente escalable.
   \item Control de cambios e historial.
   \item Orientación a micro-cambios.
   \item Desarrollo no lineal (ramas paralelas, mezcla de cambios, forks).
   \item Posibilidad de mantener múltiples repositorios remotos.
   \item Empaquetado eficiente para envío de cambios, resolución de conflictos avanzada.
  \end{itemize}

 \item Pero lleva asociado cierto coste de aprendizaje.
 \begin{itemize}
  \item ...¡que merece la pena asumir!
 \end{itemize}

 \item Integrados con IDEs populares (RStudio, Eclipse).

 \end{wideitemize}

\end{frame}

%%---------------

\begin{frame}{Documentando el proceso}
 \begin{columns}[T]
    \begin{column}{.5\textwidth}
    \includegraphics[width=1\textwidth]{figs/KnuthAtOpenContentAlliance.jpg}
%     \tiny By Flickr user Jacob Appelbaum, uploaded to en.wikipedia by users 
%     BeSherman, Duozmo (Flickr.com (via en.wikipedia)) 
%     [CC-BY-SA-2.5], via Wikimedia Commons
    \end{column}
    \begin{column}{.5\textwidth}
    \vspace*{1cm}
    \begin{quotation}
     ``I believe that the time is   ripe for significantly better documentation 
     of programs, and that we can best achieve this by considering programs 
     to be [\texttt{interactive}] works of literature''.\\
     — Donald Knuth, ``Literate Programming''. 1992.
    \end{quotation}

    \end{column}
  \end{columns}

\end{frame}

%%---------------

\begin{frame}{IPython}
\begin{itemize}
 \item Entorno de programación interactiva (incluye creación de cuadernos).
\end{itemize}

\begin{figure}
 \centering
 \includegraphics[width=0.95\textwidth]{figs/ipython.jpeg}
\end{figure}

\end{frame}

%%---------------

\begin{frame}{Conclusiones}
 \begin{wideitemize}
  \item La ciencia de datos es una mezcla de Matemáticas y Estadística, 
  ingeniería y conocimiento del área de aplicación.
  \item Elevada influencia de los aspectos tecnológicos y de implementación...
  \item … pero los otros dos factores son igual de determinantes para un 
  análisis de datos exitoso. 

 \end{wideitemize}

\end{frame}

%%---------------

\begin{frame}{Conclusiones}
 \begin{columns}[T]
    \begin{column}{.4\textwidth}
    \includegraphics[width=1\textwidth]{figs/OReilly.jpg}
    \end{column}
    \begin{column}{.6\textwidth}
    \vspace*{2.5cm}
    \begin{quotation}
     ``Data is the next Intel inside''.\\
     — Tim O'Reilly,\\
     What is Web 2.0? 2004.
    \end{quotation}

    \end{column}
  \end{columns}

\end{frame}

%%---------------

\begin{frame}{Conclusiones}
 \begin{columns}[T]
    \begin{column}{.35\textwidth}
    \includegraphics[width=1\textwidth]{figs/SherlockHolmes.jpg}
    \end{column}
    \begin{column}{.65\textwidth}
    \vspace*{1.3cm}
    \begin{quotation}
     ``I never guess. It is a capital mistake to theorize before one has data. 
     Insensibly one begins to twist facts to suit theories, instead of theories 
     to suit facts''.\\
     — Sherlock Holmes (By Sir Arthur Conan Doyle).
    \end{quotation}

    \end{column}
  \end{columns}

\end{frame}

%%---------------

\begin{frame}{Conclusiones}
 \begin{columns}[T]
    \begin{column}{.5\textwidth}
    \includegraphics[width=0.75\textwidth]{figs/questionmark.png}
    \end{column}
    \begin{column}{.5\textwidth}
    \vspace*{1.5cm}
    \hspace*{-1cm}
    \begin{quotation}
     ``If you don't know how to ask the right question, you discover nothing''.\\
     — W. Edward Deming.
    \end{quotation}

    \end{column}
  \end{columns}

\end{frame}

%%---------------

\begin{frame}{Bibliografía}
\begin{enumerate}
 \item Bell, G. et al. Beyond the data deluge. Science 323 (5919), 2009; pp. 1297-1298.
 \item Provost, F., Fawcett, T. Data Science for Business. O'Reilly Media Inc. Julio 2013.
 \item Begley, C. Glenn, and Lee M. Ellis. "Drug development: Raise standards 
 for preclinical cancer research." Nature 483.7391 (2012): 531-533.
 \item Wicherts, Jelte M., et al. "The poor availability of psychological 
 research data for reanalysis." American Psychologist 61.7 (2006): 726.
 \item Burman, Leonard E., W. Robert Reed, and James Alm. "A call for replication 
 studies." Public Finance Review 38.6 (2010): 787-793.
\end{enumerate}
\end{frame}

%%---------------

\begin{frame}{Créditos}
\begin{enumerate}
 \item Donald Knuth: Por Smallpox at it.wikipedia (Transferred from it.wikipedia) 
 [CC-BY-SA-2.0 (\url{http://creativecommons.org/licenses/by-sa/2.0)}], a 
 través de Wikimedia Commons.
 \item Tim O'Reilly: By Robert Scoble from Half Moon Bay, USA (Tim O'Reilly 
 heads panel on new advertising) [CC-BY-2.0 \url{(http://creativecommons.org/licenses/by/2.0)}], 
 via Wikimedia Commons.
 \item Sherlock Holmes: By Sidney Paget(1860-1908) [Public domain], via Wikimedia Commons.
 \item Imágenes clipart obtenidas de Openclipart, todas ellas disponibles en dominio público.
 \item Todos los logos de proyectos y/o empresas son marcas registradas, utilizados simplemente con fines ilustrativos.
\end{enumerate}
\end{frame}

%%---------------

%%---------------

\begin{frame}{Contacto}
\begin{huge}
e-mail: felipe.ortega@urjc.es\\~\\
Twitter: @jfelipe
\end{huge}
\end{frame}

%%---------------


 %% DONE
% Big data: problemas y soluciones tecnológicas
%% Big Data I: Ingeniería de datos
%
% Felipe Ortega, Javier M. Moguerza
% DEIO, URJC.
%
%
%%%%%%%%%%%%%%%%%%%%%%%%%%%%%%
\section{Big data: problemas y soluciones tecnológicas}
%%%%%%%%%%%%%%%%%%%%%%%%%%%%%%

%---------------

\begin{frame}{}
\begin{center}
 \huge Big data: problemas y soluciones tecnológicas
\end{center}
\end{frame}

%---------------

\begin{frame}{Mejor definición de big data hasta la fecha...}
\begin{center}
    \includegraphics[width=1\textwidth]{figs/Dan-Ariely-bigdata.jpeg}
\end{center}
\end{frame}

%---------------

\begin{frame}{Dimensiones big data}
\begin{wideitemize}
 \item Término controvertido, incluso para los propios profesionales.
 \item Consenso: definido por las 3 ``Vs'' [2-3].
 \begin{itemize}
  \item \textbf{Volumen} (tamaño, procesamiento).
  \item \textbf{Velocidad} (adquisición, procesamiento).
  \item \textbf{Variedad} (dimensiones).
 \end{itemize}
 \item A veces, se añade más factores (V's):
 \begin{itemize}
  \item \textit{veracidad} (integridad de datos, corrección...)
  \item \textit{valor} (el valor añadido que
 aporta big data para el negocio o dominio de aplicación).
  \item \textit{variabilidad}, \textit{visualización}, etc.
 \end{itemize}

 \item Lo importante: no es sólo una cuestión de tamaño.
\end{wideitemize}
 
\end{frame}

%%---------------

\begin{frame}{¿Cuántos son ``muchos datos''?}
\begin{wideitemize}
 \item Típicamente, más de los que podamos procesar en un sólo computador
 (incluso en un servidor muy potente).
 \begin{itemize}
  \item Por necesitar demasiada memoria.
  \item Por requerir demasiado espacio de almacenamiento.
  \item Porque no podemos almacenar el flujo de datos que nos llega de forma
  permanente (procesado \textit{streaming} vs. \textit{batch}).
  \item Porque necesitamos resultados con gran rapidez para tomar decisiones
  operativas.
 \end{itemize}
 
 \item A continuación, presentamos algunos ejemplos [4].

\end{wideitemize}
 
\end{frame}

%%---------------

\begin{frame}{Algunos números sobre big data}
 \begin{columns}[T]
    \begin{column}{.7\textwidth}
    \begin{wideitemize}
     \item \textbf{Walmart}.
     \begin{itemize}
      \item Fortune 500 Global.
      \item Mayor empleador privado del mundo (+2 millones empleados).
      \item Mayor distribuidor minorista del mundo.
     \end{itemize}

     \item Sus servidores procesan más de un millón de transacciones de clientes
     cada hora.
     \item Sus bases de datos almacenan más de 2,5 Petabytes (1 Petabyte = 1024 Terabytes).
     
    \end{wideitemize}

    \end{column}
    \begin{column}{.3\textwidth}
    \vspace*{1cm}
    \hspace*{-0.5cm}
    \includegraphics[width=1\textwidth]{figs/Walmart-exterior.jpg}
    \end{column}
  \end{columns}

\end{frame}

%%---------------

\begin{frame}{Algunos números sobre big data}
 \begin{columns}[T]
    \begin{column}{.7\textwidth}
    \begin{wideitemize}
     \item \textbf{LHC (CERN)}.
     \begin{itemize}
      \item Mayor y más potente colisionador de partículas del mundo.
      \item Una de las mayores fuentes de datos de experimentos científicos del
      mundo.
     \end{itemize}

     \item Se estima que genera unos 15 Petabytes de información anualmente.
     \item Se analizan en un sistema computacional distribuído y tolerante a fallos
     (grid computing):
     \begin{itemize}
      \item 170 centros de computación,
      \item 36 países participantes.
      \item Red global de comunicación.
     \end{itemize}

    \end{wideitemize}

    \end{column}
    \begin{column}{.3\textwidth}
    \vspace*{2cm}
    \hspace*{-0.5cm}
    \includegraphics[width=1\textwidth]{figs/CERN-LHC.jpg}
    \end{column}
  \end{columns}

\end{frame}

%%---------------

\begin{frame}{Algunos números sobre big data}
 \begin{columns}[T]
    \begin{column}{.7\textwidth}
    \begin{wideitemize}
     \item \textbf{Datos en la Web}.
     \item \textbf{Facebook} opera sobre 500 Terabytes de información de registro de actividad
     de sus usuarios, y sobre cientos de Terabytes de imágenes.
     \item Cada minuto se cargan 100 horas de vídeo en \textbf{Youtube}, y más de 135.000
     horas de vídeo son vistas.
     \item \textbf{Twitter} sirve a casi 600 millones de usuarios que generan 9.100 tweets
     cada segundo.
     \item Los sistemas de \textbf{eBay} procesan más de 100 Petabytes de información al día.

    \end{wideitemize}

    \end{column}
    \begin{column}{.3\textwidth}
    \vspace*{2cm}
    \hspace*{-0.5cm}
    \includegraphics[width=1\textwidth]{figs/Internet-map.png}
    \end{column}
  \end{columns}

\end{frame}

%%---------------

\begin{frame}{Algunos números sobre big data}
 \begin{columns}[T]
    \begin{column}{.7\textwidth}
    \begin{wideitemize}
     \item \textbf{Sector aeronáutico}.
     \item Un avión comercial de Boeing puede generar alrededor de 10 Terabytes
     de información operacional cada 30 minutos de funcionamiento.
     \item Por tanto, en un vuelo transatlántico se pueden llegar a generar
     varios cientos de Terabytes de información.
     \item Se realizan alrededor de 22.000 vuelos diarios en todo el mundo.
     \item Esto nos ofrece una idea de la ingente cantidad de datos generada
     por máquinas y redes de sensores de manera regular.

    \end{wideitemize}

    \end{column}
    \begin{column}{.3\textwidth}
    \vspace*{2cm}
    \hspace*{-0.5cm}
    \includegraphics[width=1\textwidth]{figs/Emirates-Landing.jpg}
    \end{column}
  \end{columns}

\end{frame}

%%---------------

\begin{frame}{Necesidades computacionales}
 \begin{columns}[T]
    \begin{column}{.7\textwidth}
    \begin{wideitemize}
     \item Precisamos potencia y capacidad de computación para ingeniería de
     datos.
     \item Problema: el tráfico de datos crece a mayor velocidad que nuestra
     capacidad de computación.
     \begin{itemize}
      \item (2002-2009): volúmen global del tráfico de datos se 
      multiplicó por 56; potencia de computación se multiplicó sólo por 16.
      \item (1998-2005): centros de datos crecieron en tamaño un 173\%
      anual [4], mientras que la eficiencia en consumo energético no mejoró a la par.
      \item Esto generará una enorme \textit{huella de consumo energético} para análisis
      de datos.
      \item 50\% de los centros de cómputo de datos (aprox.) solo funcionan
      al 50\% de su rendimiento máximo.
     \end{itemize}


    \end{wideitemize}

    \end{column}
    \begin{column}{.3\textwidth}
    \vspace*{1.7cm}
    \includegraphics[width=0.65\textwidth]{figs/jcartier-Cluster.png}
    \end{column}
  \end{columns}

\end{frame}

%%---------------

\begin{frame}{Tipos de datos según su estructura}

\begin{wideitemize}
  \item \textbf{Datos estructurados}: Tienen una serie de campos con significado
   predefinido. Cada campo está asociado a un tipo de datos (numérico, textual,
   doble precisión, objeto serializado...). Ejemplo: RDBMS.
   
   \item \textbf{Datos semi-estructurados}: Se representan mediante un formato
   de codificación que aporta cierta estructura e información sobre los datos
   (metadatos). Sin embargo, su contenido (número de campos, formato de cada
   campo, etc.) puede ser muy variado. Ej: documentos XML.
   
   \item \textbf{Datos no estructurados}: El formato de los datos no está claramente
   definido de forma previa. Pueden aparecer mezclados datos numéricos, textuales
   o multimedia, y en un orden imprevisible.

\end{wideitemize}

\end{frame}

%%---------------

\begin{frame}{Soluciones para datos no estructurados}
    \begin{wideitemize}
     \item Necesitamos tecnologías y métodos flexibles para gestionar este tipo de
     fuentes de datos (necesidades dinámicas e imprevisibles). Ejemplo:
     \textbf{tecnologías NoSQL}.
     \begin{itemize}
      \item Ejemplo: Esquemas \textbf{clave-valor}.
      
      \item Almacenan duplas (\textit{clave-valor}), donde las claves asociadas a
      cada valor son únicas (para acelerar las búsquedas) y los valores pueden ser
      también objetos complejos (tales como listas, tablas hash, etc).
      
      \item Ejemplo: Bases de datos \textbf{documentales}.
      
      \item Almacenan documentos representados en cierto formato de condificación
      (tales como XML, JSON o YAML).
      
      \item También siguen un esquema de almacenamiento \textit{clave-valor}, pero
      el contenido de los documentos es arbitrario, y además se ofrecen mecanismos para
      realizar búsquedas basadas en dichos contenidos (utilizando los metadatos del
      sistema de condificación).
     \end{itemize}

    \end{wideitemize}

\end{frame}

%%---------------

\begin{frame}{Tipos de procesamiento de datos}
 \begin{wideitemize}
 
 \item Clasificación de procesamiento de datos según requisitos de interacción:
 
\end{wideitemize}

\begin{enumerate}
   \item Procesamiento \textbf{batch} (también llamado \textit{offline}): No
   existen requisitos estrictos en cuanto al tiempo que podemos emplear en
   la preparación, transformación y computación de los datos almacenados.
   Ejemplo: MapReduce (Hadoop).\\
   
   \item Procesamiento de \textbf{flujos de datos} (\textit{streaming}, también
   llamado \textit{online}): Existen requisitos estrictos sobre el tiempo máximo
   que podemos emplear para preparar, trasformar y procesar los datos. Puede
   deberse a varias razones:
   \begin{itemize}
    \item Análisis interactivo.
    \item Interacción con usuarios finales (servicios, dashboards, etc.).
    \item Excesiva velocidad o volumen de datos (no podemos almacenar localmente).
   \end{itemize}


  \end{enumerate}

\end{frame}

%%---------------

\begin{frame}{Tipos de procesamiento de datos}
 \begin{columns}[T]
    \begin{column}{.5\textwidth}
    \begin{figure}
    \includegraphics[width=0.65\textwidth]{figs/conveyor-sushi.jpg} 
    \caption{Procesamiento \textit{streaming}}
    \end{figure}

    \end{column}
    \begin{column}{.5\textwidth}
    \vspace*{1.7cm}
    \begin{figure}
    \includegraphics[width=0.9\textwidth]{figs/spices-store.jpg}
    \caption{Procesamiento \textit{batch}}
    \end{figure}
    \end{column}
  \end{columns}

\end{frame}

%%---------------

\begin{frame}{Esquema procesamiento \textit{batch}}
 \includegraphics[width=\textwidth]{figs/procesamiento-batch.png}
\end{frame}

%%---------------

\begin{frame}{Esquema procesamiento \textit{streaming}}
 \includegraphics[width=0.95\textwidth]{figs/procesamiento-streaming.png}
\end{frame}

%%---------------

\begin{frame}{Tendencias procesamiento de datos}
 \begin{wideitemize}
 
 \item El procesamiento \textbf{streaming se está imponiendo} rápidamente.
 
 \item MapReduce no es suficientemente flexible ni rápido para muchos problemas
 de análisis de datos.
 
 \begin{itemize}
  \item \href{http://www.datacenterknowledge.com/archives/2014/06/25/google-dumps-mapreduce-favor-new-hyper-scale-analytics-system/}
  {Junio 2014: Google declara que dejaron de usar MapReduce hace años.}
 \end{itemize}

 \item MapReduce no es adecuado para muchos modelos de análisis de datos, que
 incluyen operaciones iterativas.
 
 \begin{itemize}
  \item Exigen un importante esfuerzo para programar estos procesos de modo que
  se reduzca el número de pasadas sobre los datos (cada iteración es muy cara).
 \end{itemize}

 \item Por contra, los sistemas de procesado \textit{streaming} se pueden adaptar
 a muchos más tipos de análisis, han sido concecibos para ser rápidos y escalables.
 
\end{wideitemize}

\end{frame}

%%---------------

\begin{frame}{Tendencias procesamiento de datos}
 
 \begin{wideitemize}
  
  \item En procesado \textit{streaming} se crean flujos de \textbf{datos inmutables}, que
  se procesan o transforman para generar nuevos flujos de datos. Se puede añadir 
  cierta \textit{persistencia}.
  
  \item También es posible combinar \textit{streaming} con procesado \textit{batch},
  (la llamada \textbf{arquitectura lambda}) pero cuidado con la duplicidad de trabajo.
  
  \item Pero exige ciertos requisitos adicionales:
  
  \begin{itemize}
   \item Sistemas de colas de mensajes / buffer de entrada que almacenen temporalmente
   los datos hasta que entren al flujo de procesado (idealmente sin péridas).
   
   \item Incluir sistemas automáticos de distribución de carga y tolerancia ante
   fallos de nodos de procesamiento.
   
  \end{itemize}

  
 \end{wideitemize}

 
\end{frame}


%%---------------

\begin{frame}{Bibliografía}
\begin{enumerate}
 \item Provost, F., Fawcett, T. Data Science for Business. O'Reilly Media Inc. Julio 2013.
 \item Cathy O'Neil, Rachel Schutt. Doing Data Science: Straight Talk from the Frontline.
 O'Reilly Media Inc. Octubre 2013.
 \item Doug Laney. 3d Data management: controlling data volume, velocity and variety.
 Appl. Delivery Strategies Meta Group (949)(2001).
 \item Kambatla, K. et al. Trends in big data analytics. Journal of Parallel and Distributed
 Computing (in press). Elsevier. Enero 2014.
\end{enumerate}
\end{frame}

%%---------------

\begin{frame}{Créditos}
\begin{enumerate}
 \item Imagen Walmart-exterior.jpg por see. CC-BY-SA-3.0, via Wikimedia Commons.
 \item Imagen inside-CERN-LHC por Juhanson. CC-BY-SA-3.0, via Wikimedia Commons.
 \item Imagen Internet map por The Opte Project. CC-BY-2.5 , via Wikimedia Commons.
 \item Imagen Boeing Emirates por Faisal Akram desde Dhaka, Bangladesh. CC-BY-SA-2.0, via Wikimedia Commons
 \item Imágenes clipart obtenidas de Openclipart, todas ellas disponibles en dominio público.
 \item Todos los logos de proyectos y/o empresas son marcas registradas, utilizados simplemente con fines ilustrativos.
\end{enumerate}
\end{frame}

%%--------------

\begin{frame}{Contacto}
\begin{huge}
e-mail: felipe.ortega@urjc.es\\~\\
Twitter: @jfelipe
\end{huge}
\end{frame}

%%---------------
 %% DONE
% Obteción de datos y formatos de representación
%%%%%%%%%%%%%%%%%%%%%%%%%%%%%%
\section{Obtención de datos}
%%%%%%%%%%%%%%%%%%%%%%%%%%%%%%

\begin{frame}{}
\begin{center}
 \huge Obtención de datos
\end{center}
\end{frame}

%%---------------

\begin{frame}{Obtención de datos}
 \begin{columns}[T]
    \begin{column}{.7\textwidth}
  \begin{wideitemize}
  \item Etapa crucial y con frecuencia infravalorada.
  \item Con frecuencia, la obtención y preparación de datos consume cerca del
  \textbf{85\% del tiempo total} del proyecto de análisis de datos.

  \item Diferentes retos.
  \begin{itemize}
   \item Multiplicidad de fuentes.
   \item Métodos de obtención de datos (\textit{scrapping}, \textit{streaming}, APIs...).
   \item Diferentes formatos de representación.
   \item Consolidación de datos obtenidos.
  \end{itemize}

 \end{wideitemize}
    \end{column}
    \begin{column}{.30\textwidth}
    \vspace*{0.8cm}
    \includegraphics[width=1\textwidth]{figs/input.png}
    \end{column}
  \end{columns}

\end{frame}

%%---------------

\begin{frame}{Obtención de datos}
 \begin{wideitemize}
  \item Multiplicidad de fuentes
\end{wideitemize}
  \begin{figure}
   \centering
   \includegraphics[width=.95\textwidth]{figs/multisource.png}
  \end{figure}

\end{frame}
\begin{frame}{Obtención de datos: aspectos de diseño}
 \begin{wideitemize}
  \item Construir módulos intercambiables para manejar cada tipo de fuente.
  \item Misma interfaz de uso, ocultando peculiaridades del manejo de cada
  fuente o tipo de datos.
  \item Considerar diseños basados en colas de elementos (datos o bloques de
  datos de entrada) que permitan gestionar:
  \begin{itemize}
   \item Distintas velocidades de adquisición.
   \item Datos heterogéneos.
   \item Mantenimiento estricto del orden de llegada.
  \end{itemize}

 \end{wideitemize}

\end{frame}

%%---------------

\begin{frame}{Obtención de datos: aspectos de diseño}
 \begin{wideitemize}
  \item Nunca debemos asumir que las fuentes nos van a enviar los datos
  correctamente representados o íntegros.
  \item Ejemplo: Beautiful Soup.
  \begin{itemize}
   \item Biblioteca Python para adquisición de datos HTML (y XML).
   \item Soporta fallos en sintaxis HTML (o XML) de los documentos de origen.
  \end{itemize}

 \end{wideitemize}

\end{frame}

%%---------------

\begin{frame}{Obtención de datos: aspectos de diseño}
 \begin{columns}[T]
    \begin{column}{.8\textwidth}
  \begin{wideitemize}
  \item Aquí importa (y mucho) la velocidad de ejecución.
  \begin{itemize}
   \item En flujos de datos en tiempo real podemos perder datos si no los recuperamos
   a tiempo.
   \item Los tiempos de espera para tratamiento de fuentes de gran volumen se pueden 
   alargar demasiado (días, semanas).
  \end{itemize}

  \item Ejemplos: lxml, UJSON (Python).
  
 \end{wideitemize}
    \end{column}
    \begin{column}{.2\textwidth}
    \vspace*{0.8cm}
    \includegraphics[width=0.8\textwidth]{figs/cronometer.png}
    \end{column}
  \end{columns}

\end{frame}

%%---------------

\begin{frame}{Obtención de datos: aspectos de diseño}
 \begin{columns}[T]
    \begin{column}{.8\textwidth}
  \begin{wideitemize}
  \item Pero también hay que respetar los límites impuestos por determiandos sistemas
  fuente.
  \begin{itemize}
   \item En APIs públicas, se suele limitar el número de consultas que pueden realizarse
   en un cierto intervalo, también la cantidad de datos devueltos por cada consulta o
   el rango temporal que podemos abarcar.
  \end{itemize}

  \item Ejemplos: \href{https://dev.twitter.com/docs/rate-limiting/1.1}{Twitter REST API 1.1}, 
  \href{https://developers.facebook.com/docs/reference/ads-api/api-rate-limiting/}{Facebook}.
  
 \end{wideitemize}
    \end{column}
    \begin{column}{.2\textwidth}
    \vspace*{0.8cm}
    \includegraphics[width=0.8\textwidth]{figs/cronometer.png}
    \end{column}
  \end{columns}

\end{frame}

%%---------------

%%%%%%%%%%%%%%%%%%%%%%%%%%%%%%
\section{Representación de datos}
%%%%%%%%%%%%%%%%%%%%%%%%%%%%%%

\begin{frame}{Representación de datos}
 \begin{columns}[T]
    \begin{column}{.75\textwidth}
  \begin{wideitemize}
  \item Formatos relacionados con tecnologías web.
  \begin{itemize}
   \item HTML, XML, JSON, YAML, etc.
  \end{itemize}

  \item Procesamiento.
  \begin{itemize}
   \item CSV, HDF5, ff, otros formatos específicos.
  \end{itemize}
  
  \item Metadatos.
  \begin{itemize}
   \item RDF (datos enlazados).
  \end{itemize}

 \end{wideitemize}
    \end{column}
    \begin{column}{.25\textwidth}
    \vspace*{0.8cm}
    \includegraphics[width=0.95\textwidth]{figs/binarydoc.png}
    \end{column}
  \end{columns}

\end{frame}

%%---------------

\begin{frame}{Representación de datos}
\begin{itemize}
 \item Ejemplos [3]: JSON
\end{itemize}

 \begin{figure}
  \centering
  \includegraphics[width=0.5\textwidth]{figs/json-sample.jpeg} 
\end{figure}

\end{frame}

%%---------------

\begin{frame}{Representación de datos}
\begin{itemize}
 \item Ejemplos [3]: YAML
\end{itemize}

 \begin{figure}
  \centering
  \includegraphics[width=0.5\textwidth]{figs/yaml-sample.jpeg} 
\end{figure}

\end{frame}

%%---------------

\begin{frame}{Representación de datos}
\begin{itemize}
 \item Ejemplos [3]: XML
\end{itemize}

 \begin{figure}
  \centering
  \includegraphics[width=1\textwidth]{figs/xml-sample.jpeg} 
\end{figure}

\end{frame}

%%---------------

\begin{frame}{Representación de datos}
\begin{wideitemize}
 \item Benchmark bibliotecas serialización (Python) [4].
\end{wideitemize}

 \begin{figure}
  \centering
  \includegraphics[width=1\textwidth]{figs/benchmark-serial.pdf} 
\end{figure}

\end{frame}

%%---------------

\begin{frame}{Representación de datos: procesamiento}
 \begin{wideitemize}
  \item Almacenamiento de estructuras de datos de gran tamaño en disco.
  
  \item Estándares
  \begin{itemize}
   \item Hyerarchical Data Format version 5 (HDF5).
  \end{itemize}
  
  \item Otros formatos específicos.
  \begin{itemize}
   \item Paquetes R \texttt{ff}, \texttt{ffbase} o \texttt{bigmemory}.
  \end{itemize}


 \end{wideitemize}

\end{frame}

%%---------------

\begin{frame}{Representación de datos: HDF5}
 \begin{wideitemize}
  \item Conjunto de datos jerárquicos, estructurados y autodescriptivos (metadatos).
  \item Capaz de escalar con facilidad al nivel de Exabyte (\textasciitilde1000 TB), compresión
  transparente, ubicación en múltiples dispositivos.
  \item Capacidad de indexación y E/S parcial.
  \begin{itemize}
   \item Evitamos cargar grandes volúmenes de datos en memoria o búsquedas secuenciales.
  \end{itemize}
  
  \item Bibliotecas disponibles en C, C++, Python, MATLAB, etc. 

 \end{wideitemize}

\end{frame}

%%---------------

\begin{frame}{Representación de datos: HDF5}
 \begin{wideitemize}
  \item Recomendable cuando los datos sean [5]:
  \begin{itemize}
   \item Grandes arrays numéricos.
   \item De tipo homogéneo.
   \item Que se puedan organizar jerárquicamente.
   \item Con metadatos de tipo arbitrario.
  \end{itemize}
  
  \item Para gestión de relaciones entre datos mejor usar bases de datos.
  \item Se puede usar también formatos más sencillos (e.g. CSV) para casos
  simples.

 \end{wideitemize}

\end{frame}

%%---------------

\begin{frame}{Representación de datos: otros formatos}
 \begin{wideitemize}
  \item Proyecto \texttt{ff} para el lenguaje R.
  \item Permite manejar grandes volúmenes de datos en R, sin necesidad de recurrir
  a clusters o cloud computing.
  \item Implementación de estructuras de datos comunes en R (ej. data frames).
  \item Implementación en C y C++ a bajo nivel, transparente para el usuario.
  \item Soporte para aplicación paralela de operaciones sobre datos en disco.
 \end{wideitemize}

\end{frame}

%%---------------

\begin{frame}{Representación de datos: RDF}
 \begin{wideitemize}
  \item Resource Description Framework.
  \item Familia de estándares de representación de metadatos promovida por W3C.
  \item Tripletas (sujeto-predicado-objeto) definen grafos dirigidos.
  \item Ofrecen información sobre ubicación y relaciones entre los datos almacenados
  (recursos web enlazados).
  \item Es posible consultar el grafo mediante el lenguaje \texttt{SPARQL}.
 \end{wideitemize}

\end{frame}

%%---------------

\begin{frame}{Representación de datos}
\begin{wideitemize}
 \item Ejemplo grafo RDF.
\end{wideitemize}

 \begin{figure}
  \centering
  \includegraphics[width=0.7\textwidth]{figs/Rdf-graph-Eric-Miller.png} 
\end{figure}

\end{frame}

%%---------------
 %% DONE
% Preparación y transformación de datos
%%%%%%%%%%%%%%%%%%%%%%%%%%%%%%
\section{Preparación y transformación de datos}
%%%%%%%%%%%%%%%%%%%%%%%%%%%%%%

\begin{frame}{Preparación y transformación de datos}
 \begin{columns}[T]
    \begin{column}{.75\textwidth}
     \begin{wideitemize}
      \item Limpieza de datos.
      \item Datos no disponibles.
      \begin{itemize}
      \item Gestión de valores vacíos.
      \item Imputación de datos no disponibles.
      \end{itemize}

      \item Transformación de datos.
      \begin{itemize}
      \item \textit{Data munging} o \textit{data wrangling}.
      \item Pasar los datos a otro formato o dejarlos preparados para luego poder 
      analizarlos más fácilmente.
      \end{itemize}

    \end{wideitemize}
    \end{column}
    \begin{column}{.25\textwidth}
    \vspace*{1cm}
    \includegraphics[width=1.2\textwidth]{figs/liftarn-Cleaning-tools.png}
    \end{column}
  \end{columns}

\end{frame}

%%---------------

\begin{frame}{Data Wrangler}
\begin{itemize}
 \item Ejemplo de este tipo de herramientas (UW Interactive Data Lab).
\end{itemize}

\begin{figure}
 \centering
 \includegraphics[width=1\textwidth]{figs/data-wrangler.jpeg}
\end{figure}

\end{frame}

%%---------------

\begin{frame}{Preparación de datos}
 \begin{columns}[T]
    \begin{column}{.75\textwidth}
     \begin{wideitemize}
      \item En primer lugar, debemos comprobar que no existen valores extraños
      ni datos omitidos.
      
      \begin{itemize}
       \item Utilizar técnicas básicas de resumen de datos.
       \item Técnicas de visualización de datos omitidos.
      \end{itemize}
      
      \item Después, tenemos dos opciones:
      \begin{itemize}
       \item Descartar los casos que contengan variables con datos omitidos.
       \item Imputar valores para los datos que faltan, utilizando técnicas
       avanzadas de imputación de múltiples valores.
      \end{itemize}

    \end{wideitemize}
    
    \end{column}
    \begin{column}{.25\textwidth}
    \vspace*{1cm}
    \includegraphics[width=1.2\textwidth]{figs/liftarn-Cleaning-tools.png}
    \end{column}
  \end{columns}

\end{frame}

%%---------------

\begin{frame}{Transformación de datos}
 \begin{columns}[T]
    \begin{column}{.75\textwidth}
     \begin{wideitemize}
      \item Otro paso crucial antes de comenzar nuestro análisis es comprobar
      la distribución de valores de los parámetros implicados.
      
      \begin{itemize}
       \item Muchas técnicas y modelos asumen que los datos siguen una cierta
       distribución (e.g. Normal), pero puede no ser cierto.
       
       \item De hecho, en la práctica nos encontramos muchas veces con distribuciones
       sesgadas (\textit{skewed distributions}) o con diferentes apuntamientos
       (\textit{kurtosis}).
      \end{itemize}
      
    \item Posibles objetivos:
    \begin{itemize}
     \item Reducir la asimetría de la distribución de valores.
     
     \item Transformar una o varias variables de forma que se parezcan más a una
     distribución Normal (univariante o multivariante).
    \end{itemize}

    \end{wideitemize}
    
    \end{column}
    \begin{column}{.25\textwidth}
    \vspace*{1cm}
    \includegraphics[width=1\textwidth]{figs/primary-transform.png}
    \end{column}
  \end{columns}

\end{frame}

%%---------------

\begin{frame}[fragile]{Aspectos adicionales}
 \begin{columns}[T]
    \begin{column}{.75\textwidth}
     \begin{wideitemize}
      \item Tranformación entre diferentes formatos de datos
      \begin{itemize}
       \item \textit{Wide format} vx. \textit{long format}.
      \end{itemize}

    \end{wideitemize}
    
    \begin{footnotesize}
    \begin{verbatim}
# Wide format
subject sex control cond1 cond2
       1   M     7.9  12.3  10.7
       2   F     6.3  10.6  11.1
       3   F     9.5  13.1  13.8
       4   M    11.5  13.4  12.9
    \end{verbatim}
    \end{footnotesize}
    
    \end{column}
    \begin{column}{.25\textwidth}
    \vspace*{1cm}
    \includegraphics[width=1\textwidth]{figs/primary-transform.png}
    \end{column}
  \end{columns}

\end{frame}

%%---------------

\begin{frame}[fragile]{Aspectos adicionales}
 \begin{columns}[T]
    \begin{column}{.75\textwidth}
        
    \begin{footnotesize}
    \begin{verbatim}
# Long format
subject sex condition measurement
       1   M   control         7.9
       1   M     cond1        12.3
       1   M     cond2        10.7
       2   F   control         6.3
       2   F     cond1        10.6
       2   F     cond2        11.1
       3   F   control         9.5
       3   F     cond1        13.1
       3   F     cond2        13.8
       4   M   control        11.5
       4   M     cond1        13.4
       4   M     cond2        12.9 
    \end{verbatim}
    \end{footnotesize}
    
    \end{column}
    \begin{column}{.25\textwidth}
    \vspace*{1cm}
    \includegraphics[width=1\textwidth]{figs/primary-transform.png}
    \end{column}
  \end{columns}

\end{frame}

%%---------------  %% TODO: IN PROGRESS
% Almacenamiento de datos: sistemas tradicionales vs. NoSQL
%%%%%%%%%%%%%%%%%%%%%%%%%%%%%%
\section{Almacenamiento de datos}
%%%%%%%%%%%%%%%%%%%%%%%%%%%%%%

\begin{frame}{Modelo de datos}
 \begin{wideitemize}
  \item Define el diseño y la implementación del sistema que almacena y gestiona
  los datos.
  \item Datos estructurados.
  \begin{itemize}
   \item Podemos almacenar sus valores en campos predefinidos con un tipo o
   una clase asociado de forma fija.
   \item Ejemplo: sistemas de bases de datos relacionales (RDBMS).
  \end{itemize}

 \end{wideitemize}

\end{frame}

%%---------------

\begin{frame}{Modelo de datos}
 \begin{wideitemize}
  \item Datos no estructurados
  \item No podemos predefinir de antemano su tipo, por lo que necesitamos un modelo
  de datos más flexible para su gestión.
  \item Relaciones complejas entre los diferentes elementos de datos.
  \item Ejemplo: Sistemas NoSQL (particionado por columnas, documentos, grafos,
  clave-valor, etc.).

 \end{wideitemize}

\end{frame}

%%---------------

\begin{frame}{Bases de datos relacionales}
 \begin{columns}[T]
    \begin{column}{.8\textwidth}
    \begin{wideitemize}
    \item Numerosas opciones en el mercado, propietarias o software libre.
    \item Larga trayectoria, tecnología muy madura y consolidada, permite predecir
    hasta cierto punto rendimientos esperados.
    \begin{itemize}
    \item Oracle, MySQL, MariaDB, PostgreSQL, SQLite, etc.
    \end{itemize}
    \item Gran variabilidad en cuanto a soporte para big data.
    \begin{itemize}
    \item Tipos de datos nativos.
    \item Particionado de tablas.
    \item Clustering, alta disponibilidad.
    \end{itemize}

    \item Object-Relational Mapping (ORM).
    \begin{itemize}
    \item Ejemplo: \href{http://www.sqlalchemy.org/}{SQLAlchemy} (Python).
    \end{itemize}

  \end{wideitemize}
    \end{column}
    \begin{column}{.2\textwidth}
    \vspace*{1.5cm}
    \includegraphics[width=0.8\textwidth]{figs/db.png}
    \end{column}
  \end{columns}

\end{frame}

%%---------------

\begin{frame}{Rendimiento RDBMS}
 \begin{columns}[T]
    \begin{column}{.8\textwidth}
    \begin{wideitemize}
    \item Es conveniente tener en cuenta algunas premisas importantes para
    analizar datos utilizando RDBMS.
    
    \item Utilizar motores \textit{ACID-compliant} únicamente cuando sea
    imprescindible.
    \begin{itemize}
     \item Si estamos trabajando con datos localmente, a los que solo accede uno o
     varios analistas, podemos usar otras opciones más rápidas.
     
     \item Ejemplos: MyISAM (MySQL), ARIA (MariaDB).
     
    \end{itemize}
    
  \item Desactivar claves primarias/foráneas en carga de datos. Después, utilizar
     solo cuando sea imprescindible.
     
  \item Activar particionado de tablas en diferentes ficheros (e.g. InnoDB).
  
  \item Cuidado con la codificación (asegurar UTF-8 siempre que sea posible).

  \end{wideitemize}
    \end{column}
    \begin{column}{.2\textwidth}
    \vspace*{1.5cm}
    \includegraphics[width=0.8\textwidth]{figs/db.png}
    \end{column}
  \end{columns}

\end{frame}

%%---------------

\begin{frame}{Rendimiento RDBMS}
 \begin{columns}[T]
    \begin{column}{.8\textwidth}
    \begin{wideitemize}
    \item La configuración por defecto de cualquier servidor de base de datos
    relacional no suele ser adecuada para el análisis de datos.
    
    \begin{itemize}
     \item Incrementar tamaño de los ficheros de claves (búsquedas, ordenación).
     
     \item Configurar adecuadamente el sistema de logging (errores, consultas
     lentas, etc.) ya que puede consumir rápidamente espacio en disco.
     
    \end{itemize}

   \item Formular las consultas basándonos siempre en conjuntos de datos y operaciones
   sobre los mismos (\texttt{JOIN}, \texttt{LEFT JOIN}, etc.).
   
   \begin{itemize}
    \item Evitar en lo posible cláusulas del tipo \texttt{WHERE X IN (...)}, ya
    que suelen involucrar búcles de búsqueda lentos.
   \end{itemize}

   \item Usar R y Python a partir de la fase de preparación
   y transformación.
   
  \end{wideitemize}
    \end{column}
    \begin{column}{.2\textwidth}
    \vspace*{1.5cm}
    \includegraphics[width=0.8\textwidth]{figs/db.png}
    \end{column}
  \end{columns}

\end{frame}

%%---------------

\begin{frame}{NoSQL}
 \begin{wideitemize}
  \item NoSQL = Not Only SQL.
  \item Escalabilidad y alto rendimiento para big data (en especial, tratamiento
  de tipos de datos muy heterogéneos o información textual).
  \item \textbf{Esquemas clave-valor}.
  \begin{itemize}
   \item Almacen de datos en pares clave-valor, no precisan esquema (Riak, Redis,
   Voldemort, etc.).
  \end{itemize}

  \item \textbf{Almacen de columnas}.
  \begin{itemize}
   \item Particionan datos por columnas, de forma que podemos paralelizar consultas
   sobre subconjuntos de datos muy grandes (HBase, Cassandra).
  \end{itemize}

 \end{wideitemize}

\end{frame}

%%---------------

\begin{frame}{Almacenamiento de datos: NoSQL}
 \begin{wideitemize}
  \item \textbf{Orientadas a documentos}.
  \begin{itemize}
   \item Cada clave está asociada a un documento, codificado según algún estándar
   de representación de datos (JSON, XML, YAML, etc.). 
   \item Los documentos pueden contener
   muchos pares clave-valor, clave-array (para listas de datos) u otros documentos.
   \item Ejemplo: MongoDB.
  \end{itemize}

  \item \textbf{Grafos}.
  \begin{itemize}
   \item Almacenan explícitamente información sobre nodos y sus relaciones, optimizando
   consultas que recorren los grafos (Neo4J).
   
   \item A cambio, tenemos que aprender un lenguaje de consultas muy diferente.
  \end{itemize}

 \end{wideitemize}

\end{frame}

%%---------------

\begin{frame}{Ventajas NoSQL}
 \begin{wideitemize}
  \item Mejor escalado horizontal, permite procesamiento paralelo.
  \begin{itemize}
   \item Consideran la agregación de nuevos recursos de computación ``en caliente''
   (autosharding, replicación).
  \end{itemize}

  \item No es necesario definir esquemas (tipos de datos), se pueden mezclar dinámicamente
  nuevos datos con los ya existentes (con limitaciones).
  
  \item Mejor integración con metodologías ágiles de desarrollo de software.
  \begin{itemize}
   \item Sprints cortos, prototipado rápido (vs. esquemas predefinidos).
   \item Dificultad para establecer a priori esquemas fijos de estructuras de datos.
  \end{itemize}

  \item Posibilidad de optimizar la recuperación de información mediante indexación
  (múltiple, dispersa, geoespacial, etc.).
  
 \end{wideitemize}

\end{frame}

%%---------------

\begin{frame}{Inconvenientes NoSQL}
 \begin{wideitemize}
  \item Requieren importantes conocimentos técnicos para su instalación, correcta
  configuración y administración (recordemos importancia del rendimiento).
  \item Todavía escasa madurez en comparación con RDBMS.
  \item Múltiples estándares de programación y APIs, incompatibles entre sí en
  muchos casos (necesidad de soporte de comunicación).
  \item Problema sistemas distribuidos: teorema CAP (\textit{consistencia}, \textit{disponibilidad},
  \textit{particionado}).
  
 \end{wideitemize}

\end{frame}

%%---------------

\begin{frame}{Ejemplo: Redis}
 \begin{wideitemize}
  \item Redis es un buen ejemplo de sistema de almacenamiento de datos NoSQL.
  
  \item Tremendamente popular, puesto que ofrece un buen rendimiento tanto con
  datos en memoria como con datos en disco.
  
  \item Buen enlace con Python: \textit{redis-py}
  \begin{itemize}
   \item \texttt{sudo pip install redis}
  \end{itemize}

  \item Ejemplo de programación con Python y Redis (Python notebook).
  
 \end{wideitemize}

\end{frame}

%%---------------

\begin{frame}{Sistemas de ficheros distribuidos}
 \begin{columns}[T]
    \begin{column}{.75\textwidth}
     \begin{wideitemize}
      \item Adecuados para distribuir datos sobre clústers de máquinas/clouds.
      \item Fuertemente ligados a tecnologías específicas.
      \item HDFS (Apache Hadoop).
      \item Google File System (GFS).
      \item Amazon S3.
    \end{wideitemize}
    \end{column}
    \begin{column}{.25\textwidth}
    \vspace*{1.5cm}
    \includegraphics[width=1.2\textwidth]{figs/cloud.png}
    \end{column}
  \end{columns}
 
\end{frame}
  %% TODO: IN PROGRESS
% Computación científica y paralelización
\include{./slides/computing}  %% TODO: IN PROGRESS
\include{./slides/communication}  %% DONE
% El paradigma MapReduce: ecosistema Hadoop
%%%%%%%%%%%%%%%%%%%%%%%%%%%%%%
\section{Ecosistema Apache Hadoop}
%%%%%%%%%%%%%%%%%%%%%%%%%%%%%%

\begin{frame}{Ecosistema Apache Hadoop}

\begin{figure}
 \centering
 \includegraphics[width=1\textwidth]{figs/hadoop-ecosystem.jpeg}
\end{figure}

\end{frame}

%%---------------

\begin{frame}{Ecosistema Apache Hadoop}

\begin{figure}
 \centering
 \includegraphics[width=1.1\textwidth]{figs/hadoop-ecosystem-full2.png}
\end{figure}

\end{frame}

%%---------------

\begin{frame}{Google File System}
\begin{figure}
 \centering
 \includegraphics[width=1\textwidth]{figs/1000px-GoogleFileSystemGFS.png}
\end{figure}

\end{frame}

%%---------------

\begin{frame}{HDFS}
 \begin{wideitemize}
  \item Modelado a partir del Google File System [2].
  \item Optimizado para maximizar el throughput, mejor cuanto más grandes
  sean los archivos.
  \item Tamaño de bloques grande, intenta optimizar distribución de datos
  en nodos siguiendo principios de localidad espacial y temporal (similar
  a jerarquía de memoria).
  \item Las primeras versiones utilizaban un nodo para metadatos (\textit{NameNode})
  y varios nodos para almacenar y trabajar con los datos (\textit{DataNodes}). En las
  versiones más modernas, datos y metadatos están distribuidos.
 \end{wideitemize}

\end{frame}

%%---------------

\begin{frame}{Almacenamiento en la nube}
 \begin{wideitemize}
  \item Es posible poder ejecutar instancias Hadoop sobre servicios de computación
  en la nube.
  \item Ejemplo: servicios web de Amazon.
  \begin{itemize}
   \item Podemos utilizar instancias de cómputo y almacenamiento de datos en Amazon
   EC2 para ejecutar nuestras tareas desde Hadoop (importación de recursos).
   \item Adicionalmente, Amazon también proporciona Amazon Elastic MapReduce (EMR), un
   entorno de gestión alternativo similar a Hadoop, que permite gestionar las tareas
   y recursos de cluster.
   \item EMR también puede albergar otros entornos para procesado de flujos de datos,
   como Apache Spark o Presto.
  \end{itemize}
  
  \item Otros proveedores: Microsoft Azure, etc.

 \end{wideitemize}

\end{frame}

%%---------------

\begin{frame}{Proyectos Hadoop: Hadoop}
 \begin{wideitemize}
  \item El núcleo de todo el ecosistema de aplicaciones.
  \item Hadoop Common Package (abstracciones) + HDFS (sistema de ficheros distribuido) +
  YARN (MapReduce engine).
  \item Se aplica el paradigma MapReduce a datos almacenados en múltiples nodos.
  \item Arquitectura Maestro-Esclavo (en su primera versión).
 \end{wideitemize}

\end{frame}

%%---------------

\begin{frame}{Proyectos Hadoop: Hadoop}
\begin{itemize}
 \item Infraestructura para Hadoop (MR1).
\end{itemize}

\begin{figure}
 \centering
 \includegraphics[width=1\textwidth]{figs/hadoop-infrastructure.png}
\end{figure}

\end{frame}

%%---------------

\begin{frame}{Proyectos Hadoop: Hadoop}
\begin{itemize}
 \item Infraestructura para Hadoop (MR1).
\end{itemize}

\begin{figure}
 \centering
 \includegraphics[width=1\textwidth]{figs/hadoop-data-manag.png}
\end{figure}

\end{frame}

%%---------------

\begin{frame}{Proyectos Hadoop: Hadoop}
 \begin{wideitemize}
  \item Gestión de tareas MapReduce (MR1).
  \item El JobTracker mantiene en memoria del nodo maestro información sobre todas
  las tareas planificadas y en ejecución.
  \item Se gestionan tanto las tareas de tipo \texttt{map} como las de tipo \texttt{reduce}, 
  asociadas a trabajos de alto nivel enviados por el cliente.
  \item Límites versión 1 (MapReduce/MR1).
  \begin{itemize}
   \item \textasciitilde 5.000 NODOS, 40.000 tareas concurrentes.
   \item Distribución fija de recursos entre procesos \textit{map} y \texttt{reduce}.
   \item Fallos acaban con trabajos en ejecución y encolados (catastrófico).
  \end{itemize}

 \end{wideitemize}

\end{frame}

%%---------------

\begin{frame}{Proyectos Hadoop: Hadoop}
 \begin{wideitemize}
  \item Principales diferencias de YARN (MR2) respecto a MapReduce (MR1).
  \begin{itemize}
   \item Soporte para mútiples estrategias (batch, flujo de datos, interactivo).
   \item Data Operating System (un solo conjunto de datos, múltiples instancias).
   \item Gestión de metadatos y trabajos distribuido.
   \item Mejor aprovechamiento de los recursos de computación y almacenamiento.
   \item Introducción de aspectos de seguridad y autenticación en el diseño.
   \item Compatibilidad hacia atrás (evitar grandes cambios para Hive, Pig, etc.).
  \end{itemize}

 \end{wideitemize}

\end{frame}

%%---------------

\begin{frame}{Proyectos Hadoop: Hadoop}
\begin{itemize}
 \item YARN/MR2.
\end{itemize}

\begin{figure}
 \centering
 \includegraphics[width=.65\textwidth]{figs/YARN-schema.jpeg}
\end{figure}

\end{frame}

%%---------------

\begin{frame}{Proyectos Hadoop: Hadoop}
 \begin{wideitemize}
  \item Principales elementos de YARN (MR2).
  
  \item \textbf{Aplicación}.
  \begin{itemize}
   \item Representación a alto nivel de un trabajo de procesamiento de datos.
   \item Puede ser un trabajo MapReduce o un script de shell.
  \end{itemize}
  
  \item \textbf{Contenedor}.
  \begin{itemize}
   \item Unidades de particionado de los recursos hardware subyacentes.
   \item Permiten la distribución de los recursos computacionales entre diferentes
   aplicaciones en ejecución, intentando maximizar el aprovechamiento de los mismos.
  \end{itemize}

 \end{wideitemize}

\end{frame}

%%---------------

\begin{frame}{Proyectos Hadoop: Hadoop}
 \begin{wideitemize}
  \item Principales componentes de YARN (MR2).
  
  \item \textbf{Resource Manager}.
  \begin{itemize}
   \item Gestión de tareas de alto nivel.
   \item Gestión de colas jerárquicas de trabajos.
  \end{itemize}
  
  \item \textbf{Application Master}.
  \begin{itemize}
   \item Uno por cada instancia a nivel de aplicación.
   \item Gestiona recursos, progreso de tareas, planificación, etc.
  \end{itemize}

  
  \item \textbf{Node Manager}.
  \begin{itemize}
   \item Agentes encargados de la gestión y monitorización de contenedores en
   cada nodo del cluster.
  \end{itemize}

 \end{wideitemize}

\end{frame}

%%---------------

\begin{frame}{Proyectos Hadoop: Hadoop}
 \begin{wideitemize}
  \item Últimos avances en YARN (MR2).
  
  \item El Resource Manager continúa siendo un punto de fallo singular. Genera
  graves trastornos en caso de pérdida de información.
  
  \item Solución: intentar reconstruir la información que contiene en caso de fallo.
  
  \item Propuesta: Zookeper (u otro meta-gestor de recursos) debe monitorizar las
  instancias de Resource Managers y actuar en caso de fallo.

 \end{wideitemize}

\end{frame}


%%---------------

\begin{frame}{Proyectos Hadoop: Apache Hbase}
 \begin{columns}[T]
    \begin{column}{.75\textwidth}
     \begin{wideitemize}
      \item Base de datos NoSQL, diseñada a imagen de Google BigTable, escrita
      en Java.
      \item Orientada a partición por columnas.
      \item Tolerancia a fallos.
      \item Compresión de datos, operaciones en memoria, filtros Bloom.
      \item Funciona sobre HDFS.

    \end{wideitemize}
    \end{column}
    \begin{column}{.25\textwidth}
    \vspace*{.7cm}
    \includegraphics[width=.9\textwidth]{figs/Apache-HBase.jpeg}
    \end{column}
  \end{columns}

\end{frame}

%%---------------

\begin{frame}{Proyectos Hadoop: Apache Cassandra}
 \begin{columns}[T]
    \begin{column}{.75\textwidth}
     \begin{wideitemize}
      \item Base de datos NoSQL liberada por Facebook en 2009.
      \item Especialmente pensada para requisitos de alta disponibilidad.
      \item Replicación en múltiples nodos (incluso alejados geográficamente).
      \item Diferentes niveles de consistencia de datos entre réplicas
      (configurable).
      \item No admite operaciones como JOIN ni subconsultas.

    \end{wideitemize}
    \end{column}
    \begin{column}{.25\textwidth}
    \vspace*{.7cm}
    \includegraphics[width=1\textwidth]{figs/Cassandra.jpeg}
    \end{column}
  \end{columns}

\end{frame}

%%---------------

\begin{frame}{Proyectos Hadoop: Apache Hive}
 \begin{columns}[T]
    \begin{column}{.75\textwidth}
     \begin{wideitemize}
      \item Sistema \textit{datawarehouse} que ejecuta sobre Hadoop.
      \item Programar operaciones para análisis de datos directamente sobre
      Hadoop puede llegar a ser muy tedioso.
      \item Hive proporciona un lenguaje de abstracción similar a las consultas
      SQL.
      \item Procesado de logs (tráfico web, sistemas), minería de texto, 
      indexación de documentos, inteligencia de negocio, predicciones y contraste
      de hipótesis.

    \end{wideitemize}
    \end{column}
    \begin{column}{.25\textwidth}
    \vspace*{.7cm}
    \includegraphics[width=1\textwidth]{figs/Apache-Hive.jpeg}
    \end{column}
  \end{columns}

\end{frame}

%%---------------

\begin{frame}{Proyectos Hadoop: Apache Pig}
 \begin{columns}[T]
    \begin{column}{.75\textwidth}
     \begin{wideitemize}
      \item Creado por Yahoo.
      \item Resuelve el problema de evitar escribir flujos de análisis de datos
      en Java para Hadoop.
      \item Pig-Latin: Lenguaje declarativo para trabajar con flujos de datos.
      \item Estrategia diferente a Hive, que está más orientado a consultas tipo SQL [7].

    \end{wideitemize}
    \end{column}
    \begin{column}{.25\textwidth}
    \vspace*{.7cm}
    \includegraphics[width=1\textwidth]{figs/Apache-Pig.jpeg}
    \end{column}
  \end{columns}

\end{frame}

%%---------------

\begin{frame}[fragile]
  \frametitle{Proyectos Hadoop: Hive vs. Pig}
  \begin{wideitemize}
      \item Cómo lo haríamos en Apache Hive [7].
  \end{wideitemize}
  \begin{verbatim}
   INSERT INTO ValuableClicksPerDMA
   SELECT dma, COUNT(*)
   FROM geoinf JOIN (
       SELECT name, ipaddr
       FROM users join clicks 
       ON (users.name = clicks.user)
       WHERE value > 0;) USING ipaddr
   GROUP BY dma;
  \end{verbatim}

\end{frame}

%%---------------

\begin{frame}[fragile]
  \frametitle{Proyectos Hadoop: Hive vs. Pig}
    \begin{wideitemize}
      \item Cómo lo haríamos en Apache Pig [5].
    \end{wideitemize}
  \fontsize{8pt}{12pt}\selectfont
  \begin{verbatim}
   Users          = load 'users' as (name, age, ipaddr);
   Clicks         = load 'clicks' as (user, url, value);
   ValuableClicks = filter Clicks by value > 0;
   UserClicks     = join Users by name, ValuableClicks by user;
   Geoinfo        = load 'geoinfo' as (ipaddr, dma);
   UserGeo        = join UserClicks by ipaddr, Geoinfo by ipaddr;
   ByDMA          = group UserGeo by dma;
   ValuableClicksPerDMA = foreach ByDMA generate group, 
   COUNT(UserGeo);
   store ValueClicksPerDMA into 'ValuableClicksPerDMA';
  \end{verbatim}

\end{frame}

%%---------------

\begin{frame}{Proyectos Hadoop: Apache Mahout}
 \begin{columns}[T]
    \begin{column}{.8\textwidth}
     \begin{wideitemize}
      \item Construcción de bibliotecas de machine learning sobre Hadoop.
      \item Clustering (K-means, K-means con lógica difusa).
      \item Sistemas de recomendación.
      \item Múltiples clasificadores:
      \begin{itemize}
       \item Regresión logística.
       \item Naive Bayes.
       \item Árboles de decisión.
       \item Random forest, etc. 
      \end{itemize}

    \end{wideitemize}
    \end{column}
    \begin{column}{.2\textwidth}
    \vspace*{.7cm}
    \includegraphics[width=1\textwidth]{figs/Apache-Mahout.jpeg}
    \end{column}
  \end{columns}

\end{frame}

%%---------------

\begin{frame}{Proyectos Hadoop: Flujos de datos}
 \begin{columns}[T]
    \begin{column}{.5\textwidth}
     \begin{wideitemize}
      \item Apache \textbf{Storm}. Framework para procesamiento de flujos de
      datos sobre HDFS, utilizando también procesado de datos en memoria.
      \item Se basa en dos tipos de componentes principales:
      \begin{itemize}
       \item Spout: Fuente de datos. Emite flujos de datos.
       \item Bolt: Procesador de datos. Puede emitir nuevos flujos.
      \end{itemize}

    \end{wideitemize}
    \end{column}
    
    \begin{column}{.5\textwidth}
    \vspace*{.7cm}
    \includegraphics[width=1\textwidth]{figs/Storm-topology.jpg}
    \end{column}
  \end{columns}

\end{frame}

%%---------------

\begin{frame}{Proyectos Hadoop: Flujos de datos}
     \begin{wideitemize}
      \item Apache \textbf{Flume}. Servicio distribuido de recolección, agregación
      y transmisión de colas de mensajes, típicamente registros (logs) de sistemas.
      \begin{itemize}
       \item Generalizable a cualquier otra fuente de datos en streaming (correo
       electrónico, redes sociales, etc.).
       \item Los ``agentes'' de Flume abstraen las labores de adquisición de datos
       de una fuente y almacenamiento de esos datos, hasta que son consumidos (por
       ejemplo, reenviándose a otro agente, o guardándolos de forma persistente en
       un sistema HDFS).
      \end{itemize}

    \end{wideitemize}

\end{frame}

%%---------------

\begin{frame}{Proyectos Hadoop: Gestión de flujos de trabajo}

     \begin{wideitemize}
      \item \textbf{Zookeper}. Gestión de procesos distribuidos.
      \begin{itemize}
       \item Registro de nombres.
       \item Monitorización y alta disponibilidad.
       \item Coordinación: información de estado, ejecución de tareas, etc.
      \end{itemize}
      
      \item \textbf{Oozie}. Planificador de tareas Hadoop.
      \begin{itemize}
       \item Funciona principalmente con flujos de tareas Hadoop y Pig.
       \item Proporciona un sistema para definir el grafo de tareas y dependencias
       entre ellas, de manera que podamos organizar la ejecución de actividades
       complejas.
      \end{itemize}

    \end{wideitemize}

\end{frame}

%%---------------

\begin{frame}{Otros proyectos: Procesamiento \textit{streaming}}
  \begin{wideitemize}
    \item Apache \textbf{Samza}: Framework para procesamiento de flujos de datos,
     concebido para su utilización junto a Apache Kafka y YARN.
    \begin{itemize}
     \item Combina un sistema de streaming de datos (Apache Kafka) con un gestor
     de recursos (que puede ser YARN) y una API que ofrece primitivas y rutinas
     de procesamiento de flujos de datos.
    \end{itemize}
    
    \item Apache \textbf{Kafka}. Sistema \textit{message broker} para gestionar 
     flujos de datos de alta velocidad y elevado throughput. Fue creado originalmente
     para tratamiento de logs de eventos en la plataforma LinkedIn, pero su uso
     se puede generalizar a otros tipos de flujos de datos.
     \begin{itemize}
     \item Puede funcionar en combinación con Samza, YARN, Spark, etc.
     \item Optimizado para procesamiento de grandes volúmenes de mensajes minimizando
     las probabilidades de pérdida de información.
    \end{itemize}
    
    \item Apache Spark.
    \begin{itemize}
     \item Lo veremos en detalle en la próxima sesión.
    \end{itemize}
      
    \end{wideitemize}

\end{frame}

%%---------------

\begin{frame}{Referencias}
 \begin{enumerate}
  \item Holmes, A. \textit{Hadoop in Practice}. Manning Publications, 2012.
  \item Google File System \url{http://research.google.com/archive/gfs.html}
  \item Murthy, C. A., Vavilapalli, V. K., Eadline, D., Niemiec, J. Markham, J.
  \textit{Apache Hadoop YARN}. Addison-Wesley Professional, Mar. 2014.
  \item Fasale, A., Kumar, N. \textit{YARN Essentials}. Packt Publishing. Feb. 2015. 
  \item  Alan Gates. Comparing Pig Latin and SQL for Constructing Data Processing 
 Pipelines.
 \url{http://developer.yahoo.com/blogs/hadoop/comparing-pig-latin-sql-constructing-data-processing-pipelines-444.html}
 \end{enumerate}

\end{frame}

 %% TODO: IN PROGRESS
% Consultas y análisis interactivo con big data
%%%%%%%%%%%%%%%%%%%%%%%%%%%%%%
\section{Análisis de flujos de datos e interactivo}
%%%%%%%%%%%%%%%%%%%%%%%%%%%%%%

\begin{frame}{Procesamiento offline vs. real-time}
 \begin{wideitemize}
  \item Un gran problema de Hadoop es que, a pesar de distribuir tareas y datos
  entre muchos nodos, puede tardar mucho.
  \item Necesidad de sistemas que puedan realizar consultas a gran velocidad
  (\textbf{interactivas}) sobre grandes volúmenes de datos.
  \item Ejemplos
  \begin{itemize}
   \item Apache Spark.
   \item Presto (Facebook).
  \end{itemize}

 \end{wideitemize}

\end{frame}

%%---------------

\begin{frame}{Apache Spark}
 \begin{columns}[T]
    \begin{column}{.8\textwidth}
     \begin{wideitemize}
      \item Framework para análisis de datos \textit{veloz}.
      \item Puede utilizar HDFS, pero no está ligado al diseño en dos fases 
      característico de MapReduce.
      \item Soporte para grafos de operaciones arbitrarios, computación en memoria
      (cuidado con requisitos del sistema).
      \item APIS: Scala, Java y Python, caché de datos en memoria, interfaces 
      para exploración interactiva de datos.
      \item Sobre el framework de análisis de flujo de datos se colocan módulos
      que ofrecen funcionalidades específicas.

    \end{wideitemize}
    \end{column}
    \begin{column}{.2\textwidth}
    \vspace*{.7cm}
    \includegraphics[width=1\textwidth]{figs/Spark-logo.png}
    \end{column}
  \end{columns}

\end{frame}

%%---------------

\begin{frame}{Stack de herramientas en Spark}
\begin{wideitemize}
 \item \url{https://amplab.cs.berkeley.edu/software/}.
\end{wideitemize}


\begin{figure}
 \centering
 \includegraphics[width=.8\textwidth]{figs/Spark-stack.jpeg}
\end{figure}

\end{frame}

%%---------------

\begin{frame}{Installing Spark}
    \begin{wideitemize}
    \item \url{http://spark.apache.org/downloads.html}.
    \item \texttt{Pre-built for Hadoop 2.4 and later}
    \item Viene con 3 entornos de programación a elegir:
    \begin{itemize}
      \item Shell Python (\texttt{./bin/pyspark}).
      \item Shell Scala (\texttt{./bin/spark-shell}).
      \item Java API (standalone app + Maven u otro builder).
    \end{itemize}

  \item Ejemplo lanzamiento shell IPython:
  \begin{itemize}
    \item \texttt{PYSPARK\_DRIVER\_PYTHON=ipython ./bin/pyspark}.
    \item \texttt{PYSPARK\_DRIVER\_PYTHON=ipython 
    PYSPARK\_DRIVER\_PYTHON\_OPTS="notebook
    -\--pylab inline" ./bin/pyspark}
  \end{itemize}

  \end{wideitemize}

\end{frame}

%%---------------

\begin{frame}[fragile]
  \frametitle{Spark en Acción (I)}
    \begin{wideitemize}
      \item Contado de palabras en Spark [3] con Python.
    \end{wideitemize}
  \fontsize{8pt}{12pt}\selectfont
  \begin{verbatim}
file = spark.textFile("hdfs://my_doc_file.txt")
 
file.flatMap(lambda line: line.split())
    .map(lambda word: (word, 1))
    .reduceByKey(lambda a, b: a+b) 
  \end{verbatim}

\end{frame}

%%---------------

\begin{frame}[fragile]
  \frametitle{Spark en Acción (II)}
    \begin{wideitemize}
      \item Regresión logística en Spark [3] con Python.
    \end{wideitemize}
  \fontsize{8pt}{12pt}\selectfont
  \begin{verbatim}
points = spark.textFile(...).map(parsePoint).cache()
w = numpy.random.ranf(size = D) # current separating plane
for i in range(ITERATIONS):
    gradient = points.map(
        lambda p: (1 / (1 + exp(-p.y*(w.dot(p.x)))) - 1) * 
    p.y * p.x).reduce(lambda a, b: a + b)
    w -= gradient
print "Final separating plane: %s" % w
  \end{verbatim}

\end{frame}

%%---------------

\begin{frame}{Arquitectura cluster en Spark}
\begin{wideitemize}
 \item \url{http://spark.apache.org/docs/latest/cluster-overview.html}.
\end{wideitemize}


\begin{figure}
 \centering
 \includegraphics[width=.8\textwidth]{figs/Spark-cluster-arch.jpeg}
\end{figure}

\end{frame}

%%---------------

\begin{frame}[fragile]
  \frametitle{Contexto en Spark}
    \begin{wideitemize}
      \item Instancia a alto nivel que controla los datos de una aplicación
      (nodos, recursos, etc.).
      \item El objeto \texttt{SparkContext} coordina las tareas en ejecución.
      \item El objeto \texttt{SparkContext} puede usar varios planificadores
      intermediarios (YARN, Mesos, etc.) para acceder a los recursos del cluster.
    \end{wideitemize}
  \fontsize{8pt}{12pt}\selectfont
  \begin{verbatim}
# Creación de un SparkContext en pyspark
from pyspark import SparkConf, SparkContext

conf = SparkConf().setMaster("local").setAppName("My App")
sc = SparkContext(conf = conf)
  \end{verbatim}

\end{frame}

%%---------------

\begin{frame}[fragile]
  \frametitle{Planificadores de tareas}
    \begin{wideitemize}
      \item Por defecto, Spark utiliza un planificador de tareas FIFO.
      \begin{itemize}
       \item Simple, no añade sobrecarga.
       \item Puede retrasar tareas que queden encoladas a la espera de que
       finalicen trabajos complicados.
      \end{itemize}

    \item Desde la versión 0.8, Spark permite configurar un planificador de tareas
    \texttt{FAIR} (similar a Round Robin).
    \begin{itemize}
     \item También admite configuración de \textit{pools} de tareas con diferente
     prioridad.
    \end{itemize}
    
    \item Desde la versión 1.2 Spark también permite asignación dinámica de recursos
   del clúster (solo YARN).

    \end{wideitemize}
  \fontsize{8pt}{12pt}\selectfont
  \begin{verbatim}
# FAIR scheduler con pyspark
val conf = new SparkConf().setMaster(...).setAppName(...)
conf.set("spark.scheduler.mode", "FAIR")
val sc = new SparkContext(conf)
  \end{verbatim}

\end{frame}

%%---------------

\begin{frame}[fragile]
  \frametitle{Resilient Distributed Datasets}
    \begin{wideitemize}
      \item Elemento central de programación en Spark.
      \item Modelan una colección de objetos \textbf{inmutable}, \textbf{distribuida}
      y \textbf{tolerante a fallos}.
      \item Los objetos que contiene se pueden distribuir en diferentes nodos
      del cluster para procesamiento en paralelo.
      \item Pueden contener cualquier objeto (Python, Scala o Java) que sea
      serializable.
    \end{wideitemize}
\end{frame}

%%---------------

\begin{frame}[fragile]
 \frametitle{Creación de RDDs}
 
 \begin{wideitemize}
  \item Dos formas de creación:
      \begin{itemize}
       \item Cargando un conjunto de datos externo.
       \item Paralelizando una colección de objetos ya existente.
      \end{itemize}

    \end{wideitemize}
  \fontsize{8pt}{12pt}\selectfont
  \begin{verbatim}
# Carga de un conjunto de datos externo
lines = sc.textFile("README.md")

# Paralelización de una colección de objetos
data = ['tokenA', 'tokenB', 'tokenC', 'tokenD', 'tokenE']
distData = sc.parallelize(data)

# Comprobamos
In [5]: type(distData)
Out[5]: pyspark.rdd.RDD

  \end{verbatim}

\end{frame}

%%---------------

\begin{frame}[fragile]
 \frametitle{Operaciones sobre RDDs}
 
 \begin{wideitemize}

   \item \textbf{Transformación}: Creación de un nuevo conjunto de datos a partir
   de otro conjunto de datos inicial.
   \item \textbf{Acción}: Devuelven un valor al programa \textit{driver} tras
   su ejecución sobre el conjunto de datos.
   
  \item \textit{Lazyness}: Los RDDs solo se computan tras la primera acción
  aplicada sobre ellos.

  \item \url{http://spark.apache.org/docs/latest/programming-guide.html}

    \end{wideitemize}
  \fontsize{8pt}{12pt}\selectfont
  \begin{verbatim}
  
lines = sc.textFile("data.txt")  ## Solo almacena puntero a datos
lineLengths = lines.map(lambda s: len(s))  ## Todavía no se computa
## Genera y distribuye trabajos
totalLength = lineLengths.reduce(lambda a, b: a + b)

lineLengths.persist()  ## Salva en memoria para uso posterior

  \end{verbatim}

\end{frame}

%%---------------

\begin{frame}[fragile]
 \frametitle{RDDs clave-valor}
 
 \begin{wideitemize}

   \item Abstracción muy útil para composición de elementos básicos en aplicaciones.
   
   \item Podemos aprovechar operaciones específicas para operar sobre este tipo
   de conjuntos de datos.
   \begin{itemize}
    \item \texttt{reduceByKey}
    \item \texttt{groupByKey}
   \end{itemize}

   \item La forma de creación difiere en cada lenguaje. En Python debemos pasar
   una función que genere tuplas de dos elementos (duplas).

    \end{wideitemize}
  \fontsize{8pt}{12pt}\selectfont
  \begin{verbatim}
# Número de ocurrencias de una línea en un archivo de texto
lines = sc.textFile("data.txt")
pairs = lines.map(lambda s: (s, 1))
counts = pairs.reduceByKey(lambda a, b: a + b)
  \end{verbatim}

\end{frame}

%%---------------

\begin{frame}{Spark Streaming}
\begin{wideitemize}
 \item \url{http://spark.apache.org/docs/latest/streaming-programming-guide.html}.
 
 \item Interfaz en Python todavía en modo experimental.
\end{wideitemize}


\begin{figure}
 \centering
 \includegraphics[width=\textwidth]{figs/SparkStreaming.jpeg}
\end{figure}

\end{frame}


%%---------------

\begin{frame}{Data Frames y Spark SQL}
\begin{wideitemize}
 \item Módulo de Spark para trabajo con datos estructurados.
 
 \item Permite tratar un esquema relacional como RDDs, en cualquiera de los tres
 lenguajes soportados por el entorno.
 
 \item Compatibilidad con numerosos estándares: tablas Hive, formato Parquet,
 datos JSON (!?).
 
 \item DataFrame: Abstracción para trabajar con tablas de un modelo relacional y
 distribuir consultas SQL en varios nodos de un cluster.
\end{wideitemize}


\end{frame}

%%---------------

\begin{frame}[fragile]
  \frametitle{Programación con Spark SQL}
\begin{wideitemize}
 \item DataFrame: Similar a una tabla en modelo relacional, equivalente al mismo
 elemento en Python (Pandas) o en R, pero incluye optimizaciones para determinadas
 operaciones.
 
 \item Se pueden crear a partir de un RDD existente, de una tabla Hive, o de
 otros orígenes de datos (formato Parquet, archivos JSON, etc.).
\end{wideitemize}

 \fontsize{8pt}{12pt}\selectfont
  \begin{verbatim}
from pyspark.sql import SQLContext
sqlContext = SQLContext(sc)

df = sqlContext.jsonFile("examples/src/main/resources/people.json")
df.printSchema()  # Imprime esquema del documento JSON
  \end{verbatim}

\end{frame}

%%---------------

\begin{frame}[fragile]
  \frametitle{Programación con Spark SQL}
\begin{wideitemize}
 \item Podemos procesar los datos como RDDs o bien como un modelo relacional con SQL.
\end{wideitemize}

 \fontsize{8pt}{12pt}\selectfont
  \begin{verbatim}
# Modo RDD
df.filter(lambda x: x.age > 21).collect()

# Modo SQL
df.registerAsTable("df_json")
result = sqlContext.sql("SELECT * FROM df_json where age > 21").
         collect()
  \end{verbatim}

\end{frame}

%%---------------

\begin{frame}{MLlib}
\begin{wideitemize}
 \item Biblioteca Spark para algoritmos de \textit{machine learning}.
 \item Incluye abstracciones para tipos de datos importantes en este contexto.
 \begin{itemize}
  \item Vectores, vectores etiquetados.
  \item Matrices locales, matrices distribuidas.
  \item Soporte para matrices densas y dispersas.
 \end{itemize}
 
 \item Numerosos algoritmos.
 \begin{itemize}
  \item Clasificación, regresión, clustering, filtrado colaborativo, reducción
  de dimensionalidad, etc.
  \item \url{https://spark.apache.org/docs/latest/mllib-guide.html}.
 \end{itemize}

\end{wideitemize}


\end{frame}

%%---------------

\begin{frame}{GraphX}
\begin{wideitemize}
 \item Procesado de grafos (también en paralelo).
 \item Introduce abstracción \texttt{Graph}: multigrafo dirigido que soporta
 propiedades para enlaces y nodos.
 \item Incluye una biblioteca de algoritmos para facilitar la construcción y el 
 análisis de grafos.
 \begin{itemize}
  \item PageRank.
  \item Aproximación de clusters (\textit{connected components}, \textit{triangle
  counting}.
 \end{itemize}
 
 \item Por el momento, la única API de programación disponible es en Scala.

\end{wideitemize}


\end{frame}

%%---------------

\begin{frame}{Presto (Facebook)}
 \begin{columns}[T]
    \begin{column}{.8\textwidth}
     \begin{wideitemize}
      \item Facebook posee uno de los almacenes de datos de mayor tamaño del
      mundo (+300 Petabytes).
      \item Necesidades: Análisis de grafos, machine learning y análisis
      interactivo.
      \item Motor de consultas SQL interactivo, enfocado en minimizar el tiempo
      de respuesta.

    \end{wideitemize}
    \end{column}
    \begin{column}{.2\textwidth}
    \vspace*{.7cm}
    \includegraphics[width=1\textwidth]{figs/presto.jpeg}
    \end{column}
  \end{columns}

\end{frame}

%%---------------

\begin{frame}{Presto (Facebook)}
\begin{itemize}
 \item Pequeña demo en línea (vídeo) [5].
\end{itemize}

\begin{figure}
 \centering
 \includegraphics[width=.8\textwidth]{figs/presto-demo.jpeg}
\end{figure}

\end{frame}

%%---------------

\begin{frame}{Referencias}
 \begin{enumerate}
  \item Karau, H., Konwwinski, A., Wendell, P., Zaharia, M. \textit{Learning Spark}.
  O'Reilly Media Inc. Feb. 2015.
  \item Karau, H., Sankar K. \textit{Fast Data Processing with Spark}. 2nd Ed.
  Packt Publishing. Mar. 2015.
  \item \url{http://spark.apache.org/examples.html}
  \item \url{https://databricks-training.s3.amazonaws.com/index.html}
  \item \url{https://prestodb.io/}
 \end{enumerate}

\end{frame}

% TODO: Gestión de datos
%%%%%%%%%%%%%%%%%%%%%%%%%%%%%%
\section{Gestión de proyectos data science}
%%%%%%%%%%%%%%%%%%%%%%%%%%%%%%

\begin{frame}{Objetivos}
  \begin{wideitemize}
  \item Presentar algunas estrategias (planificación) y tácticas (ejecución)
  que conviene tener en cuenta a la hora de realizar proyectos de análisis de
  datos, especialmente con big data.
  
  \item Modelado de datos, interpretación de resultados, advertencias contra
  malas prácticas, etc.

 \end{wideitemize}

\end{frame}

%%---------------

\begin{frame}{Teorías en busca de datos...}
  \begin{wideitemize}
  \item A veces, los investigadores o analistas pueden comenzar con una teoría
  que creen fundamentada, y buscan algún conjunto de datos que la valide [1].
  
  \item Problema: La teoría debe demostrarse para \textit{varios} (muchos)
  conjuntos de datos diferentes, no para uno solo.
  
  \item Ejemplo: tomamos muchas muestras de nuestro conjunto de datos, repitiendo
  el experimento hasta que para alguna de ellas obtengamos un resultado significativo.
  
  \begin{itemize}
   \item Según la teoría de inferencia estadística (frequentista) esto está
   garantizado.
  \end{itemize}

 \end{wideitemize}

\end{frame}

%%---------------

\begin{frame}{Y datos en busca de teorías}
  \begin{wideitemize}
  \item \textit{Data fishing} [1].
  
  \item R. H. Coase: ``If you torture the data enough, nature will always
  confess''.
  
  \item Ejemplo: Si aumentamos mucho el tamaño de nuestra muestra, siempre podremos
  llegar a obtener resultados significativos. El problema es que solo estaremos
  informando acerca de ``ruido significativo''.
  
  \item Importancia de incluir intervalos de confianza y tamaño del efecto
  (\textit{effect size}) [4].
  
  \item P-valores dependen del tamaño del efecto \textit{y} del tamaño muestral.

 \end{wideitemize}

\end{frame}

%%---------------

\begin{frame}{Sobre los p-valores}
  \begin{wideitemize}
  
  \item \textit{The earth is round ($p < 0.05$}).
  \item \url{http://mark.reid.name/blog/the-earth-is-round.html}
  
  \item Famoso artículo de J. Cohen que despeja cualquier duda sobre la correcta
  interpretación de los p-valores.
  
  \item \textbf{``el p-valor no es la probabilidad de que la hipótesis nula sea
  cierta dados los datos observados''}.

 \end{wideitemize}

\end{frame}

%%---------------

\begin{frame}{Validación cruzada}
  \begin{wideitemize}
  \item A veces el analista opta por intentar utilizar todos los datos disponibles
  para crear un modelo. Esto no suele ser una buena idea por varios motivos:
  
  \begin{enumerate}
   \item Podemos tener varios subgrupos / subpoblaciones en nuestros datos, que
   quedan enmascaradas por el conjunto.
   
   \begin{itemize}
    \item Cada subgrupo puede verse afectado por un conjunto de factores distinto.
   \end{itemize}

   \item Globalmente, los datos pueden contener mucho ruido o errores. Usando
   muestras de menor tamaño podemos controlar mejor estos aspectos.
   
  \end{enumerate}

  \item Validación cruzada: Dividimos los datos en múltiples particiones, y ejecutamos
  múltiples simulaciones del modelo usando en cada caso un conjunto diferente de
  particiones para la fase de entrenamiento (o modelado) y prueba (o validación).

 \end{wideitemize}

\end{frame}

%%---------------

\begin{frame}{Overfitting}
  \begin{wideitemize}
  \item Consiste en dedicar grandes esfuerzos para crear un modelo que se ajusta
  casi perfectamente al conjunto de datos analizado...
  
  \item Pero que, por desgracia es inservible para una nueva muestra de datos
  generada por el mismo proceso.
  
  \item Ejemplo: Podemos aumentar arbitrariamente el grado de un polinomio en un
  modelo lineal para crear curvas que pasen exactamente por todos los puntos. Sin
  embargo, el modelo no tendría validez \textit{descriptiva} ni \textit{predictiva}.

 \end{wideitemize}

\end{frame}

%%---------------

\begin{frame}{Paradoja de Simpson}
  \begin{wideitemize}
  \item Ilustrada mediante datos de matriculación en UC Berkeley.
  \item Ocurre cuando la tendencia que aparece en diferentes grupos individuales
  desaparece al combinarlos.
  
  \item Ej: parecía que las mujeres tenían menor tasa de admisiones que los hombres,
  pero se debía a que las mujeres solicitaban mayoritariamente plazas en departamentos
  con elevadas tasas de rechazo de admisión.
  
  \item \small{\url{http://en.wikipedia.org/wiki/Simpson\%27s_paradox}.}

 \end{wideitemize}

\end{frame}

%%---------------

\begin{frame}{Modelos generadores de datos}
  \begin{wideitemize}
  \item Principales ideas contenidas en [2].
  \item Los procesos (naturales o de otra índole) producen datos que podemos capturar
  para intentar estudiarlos y comprenderlos.
  \item Para ello creamos modelos que también producen datos (artificiales). Buscamos
  entonces que los datos producidor por el modelo propuesto se parezcan lo más posible
  a los datos generados por el proceso real.
  \begin{enumerate}
   \item El modelo produce datos.
   \item El modelo tiene parámetros desconocidos, que debemos estimar.
   \item Los datos del proceso real se pueden usar para reducir la incertidumbre
   sobre los parámetros desconocidos.
  \end{enumerate}

 \end{wideitemize}

\end{frame}

%%---------------

\begin{frame}{Estimación mediante función de verosimilitud}
  \begin{wideitemize}
  \item Likelihood function.
  \item Nos da una idea de la probabilidad de obtener los datos observados
  en función de los valores que adquieran los parámetros desconocidos de nuestro
  modelo.
  \item MLE: Escogemos los valores de los parámetros que maximizan la probabilidad
  de obtener los datos generados por el proceso real.
 \end{wideitemize}

\end{frame}

%%---------------

\begin{frame}{Correlaciones y causalidad}
  \begin{wideitemize}
  \item ``Correlation does not even imply correlation''. A. Gelman.
  
  \item El hecho de que encontremos correlaciones en los datos que estamos
  analizando no implica que esas correlaciones existan entre las dos poblaciones
  de interés que comparamos.
  
  \item Sin embargo, es cierto que la correlación es una de las pocas (si no la
  única) formas en las que podemos detectar causalidad.
  
  \item Importancia de los métodos experimentales para poder determinar con certeza
  relaciones causa-efecto.

 \end{wideitemize}

\end{frame}

%%---------------

\begin{frame}{Idem con series temporales}
  \begin{wideitemize}
  \item \url{http://svds.com/post/avoiding-common-mistake-time-series}.
  
  \item Basta con añadir una tendencia parecida a dos series temporales totalmente
  aleatorias para que muestren correlación entre ambas.

 \end{wideitemize}

\end{frame}

%%---------------

\begin{frame}{Correlaciones espúreas}
  \begin{wideitemize}
  \item \url{http://www.tylervigen.com/}
 \end{wideitemize}
 
 \begin{figure}
 \centering
 \includegraphics[width=.95\textwidth]{figs/spurious-corr.png}
\end{figure}


\end{frame}

%%---------------

\begin{frame}{Sesgo de grandes datos}
  \begin{wideitemize}
  \item Existe una tendencia generalizada a creer que los grandes conjuntos de
  datos pueden aportar resultados mucho más validos que conjuntos de datos más
  pequeños.
  
  \item Ejemplo: dificultades para identificar marcadores biológicos en estudios
  de medicina sobre enfermedades.

 \end{wideitemize}

\end{frame}

%%---------------

\begin{frame}{Silos de datos}
  \begin{wideitemize}
  \item Ejemplo mencionado en [3].
  
  \item En muchas organizaciones (especialmente las de gran tamaño) existen
  numerosos almacenes de datos que no están interconectados entre sí.
  
  \item A veces no se intercambian datos por puro desconocimiento, pero muchas
  otras ocasiones no se intercambian por otros motivos (privacidad, recelo, etc.).
  
  \item En ocasiones el analista puede esperar obtener más información con diferentes
  conjuntos de datos que describan el mismo proceso, pero pueden surgir muchos
  problemas:
  
  \begin{itemize}
   \item Inconsistencias.
   \item Diferencias en nomenclatura, identificadores etc.
  \end{itemize}

  \item Ejemplo: Unificación de bases de datos de compañías de telefonía móvil en España.

 \end{wideitemize}

\end{frame}

%%---------------

\begin{frame}{Referencias}
\begin{enumerate}
 \item Jules J. Berman. \textit{Principles of big data}. Morgan Kaufmann. 2013.
 \item Westfall, P., Kenning, K.S.S. \textit{Understanding Advanced Statistical Methods}.
 \item Manoochehri, M. Data Just Right. Addison-Wesley Professional. 2014.
 \item Coe, R. (2002). It's the effect size, stupid: What effect size is and why it is important.
\end{enumerate}


\end{frame}

%%---------------


%TODO:
% Referencias
%\include{references.tex}

\end{document}

